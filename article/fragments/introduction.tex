\section{Introduction}

La biométrie est un moyen d’identification reposant sur des caractéristiques propres à un individu. Cette approche présente l'avantage d'être facile d'utilisation et a le potentiel de garantir un bon niveau de sécurité. Les moyens d'authentification biométriques, les plus répondus actuellement, reconnaissent les empreintes digitales, la forme du visage ou celle de l’iris.

Au début des années 1980 il a été remarqué que, à l’instar des examples mentionnés, la façon d’utiliser un clavier est propre à l’individu. Les études menées sur le sujet présentent des diverses méthodes pour construire et distinguer ces empreintes. Néanmoins les algorithmes de différentiation présentés nécessitent la présence des échantillons d’imposteur pour l'entrainement. 

Dans les conditions d'usage réel il n'est pas envisageable de de mettre la machine à disposition d'un imposteur. De même il n'est pas de bon sens de dévoiler le mot de passe secret, nécessaire dans le cas de l'analyse statique.

Afin d’affronter ce problème, nous nous sommes posées la question comment construire un système d'authentification par dynamique de frappe au clavier en absence des données d’imposteur.


