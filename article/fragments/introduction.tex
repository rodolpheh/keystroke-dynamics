\section{Introduction}

La biométrie est un moyen d’identification reposant sur des caractéristiques propres à un individu. Cette approche présente l'avantage d'être facile d'utilisation et a le potentiel de garantir un bon niveau de sécurité. Les moyens d'authentification biométriques, les plus répondus actuellement, reconnaissent les empreintes digitales, la forme du visage ou de l’iris. Leur popularization dans le grand public est due notamment à leur emploi dans les smartphones. 

Au début des années 1980 il a été remarqué que, à l’instar des examples mentionnés, la façon d’utiliser un clavier est propre à l’individu. Les études menées sur le sujet présentent des diverses méthodes pour construire et distinguer ces empreintes. Néanmoins les algorithmes de différentiation présentés nécessitent la présence des échantillons d’imposteur pour l'entrainement. 

Dans les conditions d'usage réel il n'est pas envisageable de collecter des données d’imposteurs afin d'entrainer le modèle. La collecte des empreintes d’imposteurs implique la mise à disposition de la machine cible à l’imposteur. De plus dans le cas d’analyse statique aussi à dévoiler le mot de passe secret. 

Afin d’affronter ce problème, nous nous sommes posées la question comment construire un système d'authentification par dynamique de frappe au clavier en absence des données d’imposteur.


