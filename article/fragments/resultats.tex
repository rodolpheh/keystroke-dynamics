\section{resultats}

À partir de la base de données du GREYC, nous avons constitué, pour chaque utilisateur possédant plus de 30 enregistrements, 5 modèles. Le modèle affichant le meilleur rappel lors de la phase de pré-évaluation a été retenu et utilisé lors de la phase d'évaluation du modèle.

La figure~\ref{results} représente les FAR et FRR obtenus sur ces modèles lors de la phase d'évaluation pour chaque profil utilisateur testé.

Sur l'ensemble des modèles, on obtient un FRR moyen d'environ $0.21$ et un FAR moyen d'environ $0.10$.

Nous pouvons par exemple remarquer la précision de certains modèles tel que le modèle créé pour l'utilisateur 37 qui possède un FAR d'environ $0.024$ et un FRR d'environ $0.027$ ou le modèle créé pour l'utilisateur 85 qui possède un FAR égal à $0$ et un FRR d'environ $0.033$.

\pgfplotstableset{col sep=semicolon}

\pgfplotstableread{res/graph_data.csv}
\loadedtable

\begin{figure}
\begin{tikzpicture}
\begin{axis}[xbar=0pt,
    bar width=2pt,
    ylabel=Utilisateur,
    xlabel=Valeur,
    ytick=data,
    yticklabels from table={\loadedtable}{userId},
    height=24cm,
    xmin=0,
    xmax=1,
    ymin=0,
    ymax=98,
    width=\linewidth,
    tick label style={font=\tiny}
]

\addplot table[y=nb, x=FRR] {\loadedtable};
\addlegendentry{FRR}
\addplot table[y=nb, x=FAR] {\loadedtable};
\addlegendentry{FAR}
\end{axis}
\end{tikzpicture}
\caption{FAR et FRR pour chaque modèle obtenu}
\label{results}
\end{figure}