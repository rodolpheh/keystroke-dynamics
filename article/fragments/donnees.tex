\section{Capture des données}

La differentiation des empreintes implique l’analyse des micro-differences, de l’ordre des millisecondes. Afin de garantir une capture suffisamment precise il est nécessaire d’éliminer les retards peuvent être causées par des couches logicielles. Le langage C offre un niveau de contrôle proche de la machine en restant raisonnablement simple d’utilisation. Il nous a servi en conséquence pour capturer les événements du clavier. 

Le traitement des données, n’étant pas soumis aux même contraintes, il a pu être réalisé avec d’autres technologies. Notre choix est tombé sur Python, pour la richesse de sa bibliothèque d’utils ainsi que la facilité de prototypage.

La capture consiste à journaliser les événements clavier en sauvegardant la touche, le type d’action et le temps de l’événement. Elle est déclenchée par un programme python qui récupère également le résultat de la saisie. Afin de persister les collections d’événements, elles sont stockés sous forme d’un fichier binaire susceptible d’être élargi par des nouvelles empreintes.

Lors de la capture le logiciel demande de définir si l’empreinte provient d’un imposteur. Les empreintes marquées comme illégitimes seront ignorées durant l’entrainement du modèle. Elles sont utilisés à la fin du processus d’entraînement, pour réaliser des testes automatisées. Ces tests permettent de determiner les marqueurs de qualité et performance du modèle.

L’analyse des résultats nécessite une consistence des captures. Dans le cadre de ce travail nous avons décidés de limiter les variables d’environnement, avant d’étendre les cas d’utilisation. Afin d’éviter les variations liées aux changements du matériel, nous supposons qu’une empreinte est liée à un seul appareil et également qu’il s’agit d’un clavier matériel. Malgré le potentiel de créer un programme multi-plateforme nous nous sommes aussi restraints à utiliser le même système d’exploitation, en occurence Ubuntu 18.

Dans le journal des événements clavier on peut remarquer certains cas particuliers. Notamment la suppression de caractères. Celle-ci génère des événements non liées à la chaîne étant entrée, et nécessite de prendre en compte une remonté dans la chaîne. Une contrainte similaire se produit lors d’entrée des caractères nécessitant un appui simultané sur un ensemble de touches. Ce dernier provoque une rupture dans l’enchaînement d’appuis et de relâchements. Les deux cas ne sont pas explicitement traités lors de la capture. Ils peuvent être filtrées ou exploites afin d’enrichir le modèle.
