\section{Web-Based Benchmark for Keystroke Dynamics Biometric Systems : A Statistical Analysis\cite{giotWeb}}

\bibentry{giotWeb}\\

\subsection{Introduction}

L'auteur du document commence en nous décrivant le fonctionnement globale des recherches sur des bases de données \textit{Keystroke Dynamics}.
On se rend compte que le domaine des \textit{Keystroke Dynamics} se base globalement sur un même principe. En effet, la plupart des bases de données se reposent sur une analyse statique : la valeur d'enregistrement et de test entrée est le même pour tous les utilisateurs.\\

Dans l'optique d'adopter une approche plus actuelle de la problématique, le document se base sur des expériences effectuées sur une application web. Chaque utilisateur doit effectuer une saisie d'identifiants imposés et une saisie d'identifiants choisis par l'utilisateur. L'objectif majeur de ce document est donc de montrer la viabilité d'utiliser des identifiants personnalisés.

\subsection{Bases de données publique de dynamique de frappe}
 
Deux bases de données sont mises en avant dans cette partie, étant selon l'auteur, les bases de données avec les résultats les plus significatifs(\textit{GREYC} et \textit{DSN2009}). Afin de tenter de combler les lacunes de ces bases, sur la partie utilisateur, l'orientation choisie ici, est la variation de divers paramètres de captures, pour se rapprocher au mieux d'une expérience réaliste d'utilisation(Les variations se feront au niveau du type de clavier, des mots de passes rentrés).

\subsection{Expérimentations}

Dans un premier temps, le déroulement et les caractéristiques de l'expérience nous sont décrits. Par un envoi hebdomadaire de mail, l'expérimentation repose donc sur une base de 83 utilisateurs  et de plus de 15000 échantillons, le tout effectué sur une population d'étudiants. Les expérimentations auront donc pour but de fournir des éléments de réponses sur 4 questions :\\

\begin{itemize}
\item Différence entre mot de passe choisi et mot de passe imposé ;
\item Seuil individuel contre seuil global ;
\item Fonctionnalités à utiliser pour l'amélioration des résultats ;
\item Corrélation de la longueur, la complexité et l'entropie d'un mot de passe, par rapport aux performances de reconnaissance.\\
\end{itemize}

Il y a donc ensuite une présentations des divers formules et algorithmes afin d'observer et mesurer les divers résultats.

\subsection{Résultats}

Il s'avère qu'il n'existe pas de différence significative de fiabilité entre un mot de passe choisi et un mot de passe imposé. Le fait de croiser diverses fonctionnalités et autres informations, permet d'accroitre la fiabilité du procédé (utilisation des temps de latence et délais concernant la pression sur une touche et le relâchement d'une touche).\\

Enfin, les résultats de cette étude indiquent un impact de la taille et de l'entropie d'une mot de passe, sur les performances du système de reconnaissance. Par contre, la complexité du mot de passe n'influe en rien.

\subsection{Conclusion}

Selon les auteurs, la base de données est publique et par conséquent, accessible à tous. Il sera donc possible d'étudier les résultats de cette base, dans le cadre de nos expérimentations ultérieures. De plus, nous avons un point de vue intéressant sur l'usage d'une application web dans le cadre du domaine des \textit{ keystroke dynamics}. Enfin, cette publication apporte des formules et autres outils, et la manière d'évaluer la qualité des données d'une base de données, dans le cadre d'un système de \textit{keystroke dynamics}.
