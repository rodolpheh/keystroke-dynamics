\section{GREYC Keystroke : a Benchmark for Keystroke Dynamics Biometric Systems\cite{giotGREYC}}

\bibentry{giotGREYC}\\

\subsection{Contexte}

Cette publication est parue vingt-neuf ans après la première publication traitant de la dynamique de frappe. Selon l'état de l'art présenté par les auteurs, celle-ci est la première à réaliser une aussi grande collecte ayant en vue l'universalité des données.

\subsection{Methode experimentale}

Les auteurs se sont engagés à construire une source de données pour les futures recherches. Afin de garantir la qualité des données, la collecte a été réalisée selon les critères pertinents relevés par les études préalables.\\

Les auteurs ont réalisé un logiciel permettant de saisir des échantillons de façon reproductible. Il s'agit des échantillons de type statique, tels que les mots de passe.\\

Pour construire le lot, 133 personnes ont participé en entrant entre 5 à 107 échantillons. En quatre mois, 7555 captures ont été réalisées.\\

Lors des essais les sujets ont été amenés à changer de matériel et espacer les saisies dans le temps.\\

Les sujets participants étaient en majeure partie masculins agés de 18 à 25 ans.

\subsection{Résultat}

Ce travail se concentre sur la construction des matériaux pour les futures analyses.\\

Les données réunies sont disponibles sous forme brute et traitée. Sous leur forme brute, il s'agit de la touche qui a été pressée, associée d'un timestamp. Sous leur forme traitée, nous obtenons les temps de pression et de relâchement ainsi que le temps entre une pression et un relâchement et entre un relâchement et une pression. Le logiciel permet d'enrichir le lot à condition de partager des résultats.\\

Comme il s'agit d'une analyse statique des mots de passe, on peut parler d'essais corrects ou erroné. Les auteurs ont identifié six facteurs causant des erreurs lors de la saisie :\\

\begin{itemize}
  \item Le mot de passe est trop long ;
  \item L'utilisateur n'est pas habitué au clavier et hésite ;
  \item L'utilisateur veut taper plus vite qu'il n'est capable de le faire ;
  \item L'utilisateur a oublié le mot de passe ;
  \item L'utilisateur est perturbé par l'environnement ;
  \item L'utilisateur doit taper un mot de passe prédéfini.
\end{itemize}

\subsection{Conclusion}

Du fait de la complexité d'installer un protocole d'acquisition des données, il est conseillé de se baser sur le lot de données existant plutôt que de construire un nouveau. Le choix des méthodes est argumenté et les conditions de collecte sont documentées. De cette façon les résultats obtenus pourront être comparés avec d'autres études.

Pour notre travail, cela nous apporte une base de données qui nous permettra d'évaluer nos résultats comparativement à ceux d'autres travaux utilisant les mêmes données, dans une logique de \textit{benchmark}. Cela nous permet aussi de ne pas mobiliser trop de notre temps dans l'acquisition de données en grande quantité, tout en rendant nos propres résultats plus facilement reproductibles.