% WELCOME WELCOME WELCOME

% Bienvenue dans notre rapport bibliographique en LaTeX

% Les paquets importés et les paramêtres de ceux-ci peuvent se trouver dans headers.tex

% La bibliographie peut se trouver dans RapportBibliographique.bib

% Les synthèses individuelles sont à mettre dans 'papers'
% Elles peuvent être inclus avec la commande \includesynthese{filename}
% Pas besoin d'indiquer l'extension, LaTeX va chercher automatiquement pour des *.tex

% Pour faire une citation, il faut utiliser la commande \cite{reference}
% La reference peut être trouvée dans le fichier de bibliographie

\documentclass[a4paper,11pt]{article}
\usepackage[T1]{fontenc}
\usepackage[utf8]{inputenc}
\usepackage{lmodern}
\usepackage[francais]{babel}
\usepackage{color}
\usepackage{hyperref}
\usepackage{cite} 

\hypersetup{
    colorlinks=true, % make the links colored
    linkcolor=blue, % color TOC links in blue
    urlcolor=blue, % color URLs in red
    linktoc=all % 'all' will create links for everything in the TOC
}

\newcommand\includesynthese[1]{\include{./papers/#1}}


% Ici on indique le titre et les auteurs (\\ pour passer une ligne dans la liste d'auteurs)

\title{Rapport bibliographique}
\author{Rémi Bourgeon\\
Franciszek Dobrowolski\\
Rodolphe Houdas\\
Timothé Jouglet}

% Let's start that shit

\begin{document}

% On ajoute la page de garde et la page de sommaire

\maketitle
\newpage
\tableofcontents
\newpage

% ABSTRACT

\begin{abstract}
Dans ce document, nous présenterons la dynamique de clavier et son utilisation dans le cadre de l'authentification par biométrie. Dans une première partie du document sont présentées les synthèses individuelles des publications retenues durant notre travail de documentation. Une synthèse globale vient ensuite reprendre les points principaux de ces publications dans un état de l'art du domaine.
\end{abstract}

\newpage

% Le lexique se situe dans Lexique.tex

\section{Lexique}

\begin{description}
  \item[DDF (Dynamique De Frappe)] Rythme de frappe au clavier
  \item[DET (Detection Error Tradeoff)] Courbe traçant le taux de faux négatifs (FRR) en fonction du taux de faux positifs (FAR).
  \item[EER (Equal Error Rate)] Mesure utilisée en biométrie lorsque le FAR et le FRR sont égaux.
  \item[Enregistrement (\textit{enrollment})] Démarche à suivre par un utilisateur pour s'enregistrer dans un système biométrique
  \item[FAR (False Acceptance Rate)] Taux de faux positifs, c'est-à-dire quand l'algorithme accepte un imposteur.
  \item[FRR (False Rejectance Rate)] Taux de faux négatifs, c'est-à-dire quand l'algorithme refuse un individu légitime .
  \item[FTA (Failure To Acquire)] Échec de la capture. Dans le cas de la dynamique de frappe, il peut s'agir d'une mauvaise saisie.
  \item[FTE (Failure To Enroll)] Échec de l'enregistrement d'un nouvel utilisateur, principalement dû à un FTA.
  \item[\textit{Keystroke Dynamics}] champ de la biométrie comportementale qui utilise les données de frappe au clavier pour reconnaître un utilisateur.
  \item[ROC (Receiver Operating Characteristic)] Courbe traçant le taux de vrais positifs en fonction du taux de faux positifs.
  \item[\textit{Timestamp}] Donnée temporelle permettant d'identifier une date avec une précision variant entre la nanoseconde et la milliseconde suivant le format.
\end{description}


% On ajoute les synthèses individuelles des publications

\includesynthese{Monrose} %1997
\includesynthese{Peacock} %2004
\includesynthese{Buchoux} %2008
\includesynthese{Hu} %2008en \textit{keystroke dynamics}
\includesynthese{Rosenberger} %2009
\includesynthese{Giot_Benchmark} %2009
\includesynthese{Killourhy_Maxion} %2009
\includesynthese{GiotSVM} %2009
\includesynthese{Giot_Web} %2012
\includesynthese{Panasiuk} %2016

% Synthèse globale

\section{Introduction}

Notre société moderne a été transformée par la démocratisation des outils informatiques. La transformation numérique du travail est un des défis principaux posés aux entreprises d'aujourd'hui, et la vie quotidienne de milliards d'individus dépend de l'accès à des services numériques en ligne, et les profils associés.

L'accès aux données confidentielles stockées dans des ressources informatiques est devenu tellement central que les enjeux de sécurité et de protection de la vie privée se retrouvent partout. Le RGPD, la dernière législation européenne sur la protection des données personnelles, atteste de l'urgence du sujet.

Pour protéger les données, les solutions qui ont toujours accompagné l'informatique sont systèmes permettant de \textbf{prouver son identité}. Le moyen le plus répandu et auquel le public s'est habitué est de retenir un mot de passe et/ou un nom d'utilisateur. Cependant, de nombreuses failles de sécurité ont montré que ces systèmes avaient leur limites. En effet, un mot de passe complexe à retenir ne garanti pas qu'il soit difficile à trouver par une machine. De plus, la complexité de ce mode d'authentification augmente avec l'accroissement du nombre de mots de passe différents à retenir. Ainsi, la plupart des failles de cette méthode d'authentification sont humains :

\begin{itemize}
	\item il est difficile de retenir autant de mots de passe que ce qu'il est nécessaire pour mener une vie quotidienne normale ;
	\item la solution la plus couramment adoptée est la réutilisation de mot de passe ;
	\item la répétition des mots de passe et la multiplication des endroits où ils sont stockés sont idéals pour les attaquants qui sont assistés de programmes beaucoup plus doués que les humains pour deviner des mots de passe et les retenir.
\end{itemize}

Par conséquent, la recherche et l'industrie ont tenté de créer différents moyens d'authentification qui seraient attaché à l'individus plutôt que d'être conditionnés à leur capacité à choisir des mots de passe complexe et à ne pas les réutiliser. Les \textit{keystroke dynamics} font partie de ce groupe de solutions.

Comme le champ de la recherche en la matière est fortement dépendant de concepts qui lui sont propre, nous commencerons par une partie de définitions. Nous examinerons ce qui vient de la biométrie elle-même, ce qui est propre aux \textit{keystroke dynamics} et ce qui est apporté par le \textit{machine learning}.

Par la suite, nous examinerons les enjeux d'acquisition des données, un point crucial pour la recherche en biométrie.

Enfin nous examinerons les différentes méthodes de classification utilisées, et les métriques qui permettent l'évaluation des modèles développés.

\section{Définitions}

\subsection{Biométrie}

Wood\cite{wood1977} définie trois approches pour identifier une personne :\\

\begin{itemize}
	\item utiliser ce que l'utilisateur \textbf{sait}, c'est l'approche par mot de passe ;
	\item utiliser ce que l'utilisateur \textbf{a}, c'est l'approche retenue par les clés physiques et les systèmes double authentification par SMS ;
	\item utiliser ce que l'utilisateur \textbf{est}, c'est l'approche de la biométrie.\\
\end{itemize}

Un système de contrôle biométrique est un système automatique de mesure basé sur la reconnaissance de caractéristiques propres à l'individu. Il s'agit de l'authentification basée sur les caractéristiques physiques ou comportementales. C'est à dire ce que l'individu est ou ce qu'il sait faire.\\

Tous les systèmes biométriques génèrent, à partir de caractéristiques physiques ou comportementales, une signature qui sera ensuite comparée au modèle enregistré. Cette comparaison nous permet d'obtenir un degré de ressemblance avec le modèle enregistré. En fonction d'un seuil, défini pour que le système soit conforme à des standards de sécurité, on valide ou non l'utilisateur.

\subsubsection{L'analyse morphologique}

L'analyse des caractéristiques physiques est aussi appelée l'analyse morphologique et être appliquée sur les empreintes digitales, l'iris, la morphologie de la main ou encore les traits du visage.\\

L'analyse des empreintes digitales consiste à scanner le dessin formé par les lignes de la peau des doigts d'un individu. Les empreintes ont l'avantage d'être uniques et pratiquement immuables au cours de la vie.\\

Pour scanner l'iris, une caméra acquiert son dessin et le compare à un fichier d'identification contenant le modèle de référence stocké pour un individu.\\

La morphologie de la main est particulièrement populaire aux États-Unis. Elle consiste à mesurer des caractéristiques de la main comme la longueur et la largeur des doigts ou la forme des articulations. Le désavantage de cette méthode est la similitude des résultats pour les personnes venant de la même famille, ayant une morphologie semblable.\\

L'analyse du visage se base sur la mesure des distances entre les éléments stratégiques du visage. On analyse l'écartement des yeux,  les arêtes du nez, les commissures des lèvres, les oreilles et le menton. Les systèmes de reconnaissance du visage sont actuellement en évolution et permettent de prendre des échantillons en mouvement ou prendre en considération le vieillissement d'un individu.

\subsubsection{L'analyse comportementale}

L'analyse comportementale se concentre sur ce que l'individu sait faire de manière unique. Celle-ci peut reposer sur l'analyse de la voix, la dynamique des signatures ou la dynamique de frappes au clavier.\\

L'analyse de la voix consiste à mesurer les intonations qui sont ensuite comparées à l'empreinte pour confirmer l'identité. Ce système est utilisé pour accéder à certains service bancaires ou des lignes de service après-vente automatisées.\\

La dynamique de signature se concentre sur les gestes effectués lors de la signature. Cette analyse considère la direction et la pression d'un tracé du stylo et combine ces informations à la forme de la signature afin de déterminer l'identité.\\

La dynamique de frappe au clavier repose sur un principe similaire, mais au lieu d'observer les gestes, on mesure les temps de pression et de relâchement des différentes touches\cite{giotGREYC} ou l'ordre des n-graphes qui sont tapés\cite{bergadano2002,gunetti2005}.

\subsubsection{Analyse des performances}

Pour évaluer l'efficacité des différents systèmes biométriques nous pouvons considérer trois facteurs majeur :\\

\begin{itemize}
\item la performance - analyse des données statistiques sur le performance du système, comme EER, FTE, FTA ou temps de traitement
\item l'acceptabilité - les informations sur la perception et l'acceptation par l'utilisateur 
\item sécurité - quantifie la sécurité du système, par example combien de fraudes un imposteur peut utiliser\\
\end{itemize}

Ces trois caractéristiques doivent être prises en compte simultanément pour juger la performance d'un système biométrique. Un niveau bas de fausses identifications ne pourra pas être significatif si le taux de refus des utilisateurs légitimes est élevé. Dans se cas la solution sera condamnée à ne pas être utilisée.
>>>>>>> Stashed changes

\subsubsection{Conclusion}

Les méthodes d'authentification biométriques présentent des avantages de fiabilité, rapidité et facilité d'utilisation. Certains domaines peuvent encore être approfondis, notamment la dynamique de la frappe au clavier.

\subsection{Dynamique de frappe}

Les \textit{keystroke dynamics}, ou en français la \textbf{dynamique de frappe au clavier} est une méthode biométrique comportementale. Depuis le début de la recherche dans ce domaine\cite{rand}, l'intérêt de cette méthode a été de proposer une identification peu coûteuse en matériel et en traitement \cite{monrose1997}. La donnée en la matière est théoriquement abondante, et l'acquisition des données est transparente pour les utilisateurs. La littérature distingue deux types de systèmes biométriques utilisant la dynamique de frappe.\\

%\paragraph{Statique}
Identification d'un utilisateur en fonction d'un secret connu à l'avance par le modèle.
%\paragraph{Dynamique}
Identification de l'utilisateur sur la base d'un texte libre.

\begin{description}
  \item[Statique] Il s'agit de l'identification ou de l'authentification d'un utilisateur à l'aide d'un mot de passe fixe choisi par l'utilisateur (ou l'expérimentateur dans le cas des recherches). Cela permet par exemple de faire de l'authentification bi-factorielle combinant un mot de passe et la dynamique de frappe.
  \item[Dynamique] Il s'agit de l'identification ou de l'authentification d'un utilisateur sur du texte libre. Cela permet par exemple de faire du \textit{monitoring} de l'identité de l'utilisateur et de verrouiller le système lorsqu'un utilisateur non autorisé est détecté.\\
\end{description}

La dynamique de frappe au clavier va se baser sur une multitude de grandeurs mesurées pour constituer un profil qui soit unique par utilisateur\cite{giotThese,Hu2008,gunetti2005,bergadano2002}.


\section{Acquisition}
Plusieurs auteurs soulignent l'importance de l'acquisition des données. Pour évaluer la pertinence des acquisitions, on peut se servir des métriques établies à ce propos \cite{giotWeb}.\\
% à développer pour faire une transition
Afin de constituer un module d'acquisition des données robuste et fiable, il faut réfléchir au protocole de captation mis en place. Par ailleurs, une fois la donnée brute enregistrée, il faut déterminer comment la traiter pour qu'elle soit exploitable.

\subsection{Protocole d'acquisition}

Afin d'avoir un système "\textit{Keystroke Dynamics}" efficace et fiable (c'est à dire laissant entrer les bon utilisateurs lors de l'analyse et restreignant l'accès à des imposteurs), il est indispensable de traiter et filtrer la donnée en amont. Il est également nécessaire d'établir des critères et de filtres afin de  limiter les erreurs. Nous allons donc ici nous intéresser aux critères à regarder afin d'obtenir une donnée pertinente.

\subsubsection{Grandeurs mesurées}
La dynamique de frappe au clavier va se baser sur une multitude de grandeurs mesurées pour constituer un profil qui soit unique par utilisateur\cite{giotThese,Hu2008,gunetti2005,bergadano2002}.

Dans les données qu'on peut chercher à capter, on peut citer :

\begin{itemize}
	\item[\textit{Press event}] : événement de clavier associant un \textit{timestamp} et un code de touche de clavier, permettant d'indentifier l'instant auquel l'utilisateur a commencé à appuyer sur la touche.
	\item[\textit{Release event}] : même type d'événement pour l'instant auquel l'utilisateur a relaché la touche.
	\item[Pression] : les travaux portant sur l'adaptation des \textit{keystrokes dynamics} à des smartphones vont parfois proposer la donnée concernant la pression de la frappe sur le clavier tactile.
\end{itemize}

Une fois ces données \textbf{brutes} captées, le protocole implique souvent de calculer des données dérivées (cf. \ref{subsec:featureengineering}).

\subsubsection{Qualité de la donnée}

Un des débats revenant le plus souvent est l'acceptation de l'erreur lors de l'acquisition de la donnée. En effet, selon le document de Romain Giot sur les benchmark\cite{giotBenchmark}, la correction d'une erreur change la manière de taper un texte, vu qu'il faut  prendre en compte les \textit{inputs} nécessaire pour rectifier le texte erroné. Par conséquent, la majorité des base de données du domaine ne permettent pas l'erreur lors de l'acquisition: l'utilisateur se voit donc forcer de reprendre l'acquisition de zéro en cas d'erreur. Néanmoins, cette particularité reste un champ de recherche possible pour l'amélioration de la technologie.\\

Une autre question récurrente dans le domaine des \textit{Keystroke Dynamics} concerne les mots de passes : doit-on forcement utiliser un unique mot de passe/une unique paraphrase pour tous les utilisateurs de la phase d'acquisition ? Dans la majorité des base de données, une paraphrase ou un mot de passe unique est pré-défini et utilisée par les utilisateurs. Cette méthode a l'avantage d'être peu coûteuse en terme de temps et moins complexe à gérer au niveau des bases de données (l'information de réussite dans la frappe de cette paraphrase est une valeur booléenne). Néanmoins cette pratique reste peu réaliste par rapport à l'utilisation que l'on voudrait en faire (permettre une double authentification, afin d'accroitre la sécurité lors d'une connexion). De plus, l'utilisateur doit donc apprendre une chaine de caractères ce qui prend du temps et entraine une évolution de la frappe au fil des acquisitions.\\

Des tests ont cependant été effectués par le laboratoire du Greyc\cite{giotWeb} par le biais d'une application web (Un contexte plus réaliste à la mise en pratique de cette technologie). Il s'avère par le biais de ces expériences comparatives, qu'il n'y ait pas de différence notable de fiabilité et de performance entre un mot de passe unique et un mot de passe choisi par l'utilisateur.\\

En conclusion de cela, il n'est pas nécessaire d'utiliser un mot de passe unique pour les tests, mais reste fortement recommandé dans une souci de simplicité.\\

Les données de tests doivent néanmoins se plier à certains critères précis :\\

\begin{itemize}
\item L'entropie de la donnée : L'entropie d'une donnée / d'un mot de passe correspond à la capacité de ne pas prédire la chaine de caractères composant la donnée. L'entropie ne concerne pas forcement la variété des caractères utilisés (il s'agit plus d'une notion de complexité dans ce cas) mais plus le fait d'eviter les chaines simples/avec du sens (par exemple avec la chaine "azertyuiop"). L'entropie a une influence sur les performances d'un système \textit{Keystroke Dynamics}(\cite{giotWeb}) : la valeur d'\textit{ERR} (Equal Error Rate) est plus importante si l'information d'entropie est croisée.\\

\item La taille de la donnée : La taille de la donnée/du mot de passe, a également un impact sur la performance d'un system \textit{KD}.\\

\item La complexité de la donnée : la complexité de la donnée / d'un mot de passe est un indicateur de sécurité. Plus un mot de passe est complexe (utilisation de caractères spéciaux, enchainement non logique de caractères, etc...), plus le mot de passe sera dur à casser. Selon Romain Giot (\cite{giotWeb}), la complexité du mot de passe n'influe en rien sur les performances d'un système de \textit{Keystroke Dynamics}. Il s'agit plus d'un paramètre de sécurité utile pour bloquer le cassage de mot de passe.\\

\end{itemize}

En vue de ces paramètres, il est plus que nécessaire de bien calibrer et sélectionner sa donnée (mot de passe) afin d'optimiser la performance de la base et du système \textit{Keystroke Dynamics}\\

\subsubsection{Critères propre à l'utilisateur et à l'environnement d'acquisition}

% Y a quoi là normalement ? Exemple ?

\subsubsection{Importance des bases de données publiques}

Romain Giot \textit{et al.}\cite{giotGREYC} soulignent la difficulté de constituer une base de données conséquente pour la recherche en \textit{keystroke dynamics}. En effet, de nombreux paramètres sont susceptibles d'ajouter du bruit dans la donnée captée, voire de la rendre inutilisables. Cependant la littérature n'est pas unanime quant au fait que des erreurs de saisie viendraient renforcer la fiabilité des modèles ou la diminuer.\\

Romain Giot \textit{et al.} \cite{giotGREYC} ainsi que Killourhy et Maxion \cite{killourhy2009} ont donc tenté de répondre à cette problématique en constituant des bases de données conséquentes, basées sur un protocole d'acquisition et un formatage documentés.\\

Ces travaux nous permettent de nous insérer dans un ensemble de publications réutilisant ces données pour rendre les résultats comparables, ce qui n'était pas vraiment possible avant 2009.\\

Cependant, il est tout de même utile de souligner qu'il reste encore beaucoup à faire en la matière, notamment en comparaison des autres champs d'étude du \textit{Machine Learning}, qui disposent de bases de données de millions d'images pour la reconnaissance des visages, par exemple.\\

De plus, et afin de mettre en place un protocole d'acquisition de la donnée dans une application concrète, il faut avoir à l'esprit les enjeux soulignés lors de la constitution de la base de données GREYC\cite{giotGREYC}.\\

\subsection{Ingénierie des caractéristiques}
\label{subsec:featureengineering}

Dans sa thèse \cite{giotThese}, Romain Giot fait l'inventaire des méthodes utilisées dans le domaine des \textit{keystroke dynamics} pour préparer la donnée avant de l'utiliser dans un modèle. Diverses méthodologies peuvent être utilisées pour formaliser les données. On a des exemples de calculs de moyennes de certaines caractéristiques de la donnée. D'autres proposent la constructions de \textit{timing vectors}\cite{killourhy2009}, soit la constitution d'un vecteur multidimensionnel dont les coordonnées correspondent à des données temporelles qui décrivent la rythmique de la frappe de l'individu.\\

Certains des algorithmes de \textit{machine learning} utilisés procèdent eux-mêmes à un traitement préalable des données pour affiner la discrimination entre les différentes classes à discriminer.

Mais d'une manière générale, les traitements les plus courants sur les données de frappe au clavier sont le calcul de données de frappes \textbf{calculées} depuis la donnée brute :

\begin{description}
	\item[\textit{Seek/Flight time}] : le temps entre le relâchement d'une touche et le début de la pression sur une autre touche ;
	\item[\textit{Hold/Dwell time}] : le temps entre le début de la pression d'une touche et son relâchement. \cite{kobojekRNN}
\end{description}

Souvent les données sont aussi traitées pour l'analyse des digraphes et des trigraphes (généralisable aux n-graphes \cite{Hu2008}). Il s'agit de l'étude des \textit{timings} sur des groupes de deux, trois ou n caractères. Cette technique intervient beaucoup dans la \textbf{méthodologie dynamique} \ref{subsec:ddf}. Dans ce contexte les chercheurs ont tendance à sélectionner les n-graphes au préalable en fonction de leur importance. Par exemple, le trigraphe "\textit{-ing}" sera plus intéressant à examiner pour un utilisateur anglophone que francophone du fait de la fréquence de cette séquence de lettres en Anglais.

Après avoir calculé toutes les caractéristiques jugées nécessaires au traitement, l'échantillon est mis en forme. En \textit{machine learning} cela prend la forme d'un \textbf{vecteur}, soit les coordonnées d'un point en dimension n, avec n le nombre de caractéristiques retenues, brutes ou dérivées.

\section{Classification et Évaluation}

Une fois l'acquisition des données utilisateur réalisée, il reste à produire un système de vérification capable de discriminer un imposteur par rapport à un ou plusieurs utilisateurs authentiques. Les algorithmes et méthodologies utilisées varient en fonction des objectifs recherchés.

\subsection{Classification}
\subsubsection{Méthodologies}
La première étape est de construire un modèle de vérification en classant les données disponibles en plusieurs "classe" de profils. Plusieurs méthodes possibles existent pour classer les données :

\begin{itemize}
	\item Méthodes statistiques ;
	\item Algorithmes de \textit{Machine Learning} \bibentry{Hu2008};
%	\item Réseaux de neurones. À proprement parler il s'agit aussi de \textit{Machine Learning} mais la littérature a tendance à les séparer du reste car ils ont des propriétés propres et requièrent des performances bien plus importantes que les algorithmes classiques.
\end{itemize}

Parmis les algorithmes à base de \textit{machine learning} aperçus dans la littérature, nous pouvons citer :

\begin{description}
  \item[kNN\cite{Hu2008} (\textit{k-Nearest Neighbors})] Il s'agit d'un algorithme de classification qui mesure la distance entre une nouvelle donnée et les données précedemment enregistrées. On sélectionne alors les k plus proches voisins et on fait la moyenne de leur classe pour déterminer la classe de la nouvelle donnée (si sur 3 voisins plus proches, 2 d'entre eux sont de la classe 2 et un de la classe 1, alors la nouvelle donnée fait partie de la classe 2).
  \item[SVM\cite{giotSVM} (\textit{State Vector Machine})] Il s'agit d'un algorithme de classification, généralisation des classifieurs linéaires, dont les particularités sont la transformation de l'espace de représentation lorsque les données ne sont pas linéairement séparables et la maximisation de la marge au niveau des échantillons les plus proches de la frontière de décision afin de redéfinir une frontière de décision plus précise.
  \item[ANN (\textit{Artificial Neural Network})] Il s'agit d'un algorithme comprenant une couche de neurones d'entrée, une couche de neurones de sortie et une ou plusieurs couches de neurones intermédiaires. Les neurones intermédiaires agissent comme des coefficients qui sont appliqués dans une équation dont les paramètres d'entrée sont les neurones d'entrée et la sortie, la classe correspondante aux données entrées. Sa construction se révèle très simple et permet de résoudre des problèmes très complexes. Son efficacité sur des jeux de données comprenant un grand nombre d'attributs (> 50) en fait un outil de choix pour la discrimination de schémas complexes tels que des enregistrements de frappe au clavier. Il a comme désavantage d'être très gourmand en mémoire et en ressources processeurs, tant pour l'entraînement que pour la prédiction.
  \item[One-class SVM\cite{oneclassSVM}] Il s'agit d'une SVM non-supervisée. Elle est utilisée pour des tâches d'\textit{outlier detection}, c'est-à-dire la détection de données anormales, n'appartenant pas au groupe de données initial. Pour être utilisée, elle ne nécessite qu'un groupe de données appartenant à une seule classe. Elle est donc parfaitement adaptée à la détection d'un seul utilisateur en absence de données d'imposteurs.
\end{description}

Les algorithmes de \textit{machine learning} supervisés qui font de la classification multi-classe (c'est-à-dire une classification entre trois classes ou plus) sont divisés en deux types :

%http://scikit-learn.org/stable/modules/svm.html#multi-class-classification

\begin{description}
	\item[\textit{one-vs-one}] : pour chaque couple de classe, on crée un classifieur binaire. Pour classifier une nouvelle donnée, on effectue une prédiction avec tous les classifieurs et on récupère la classe ayant été la plus prédite parmis tous les classifieurs.
	\item[\textit{one-vs-rest}] : pour chaque classe, on crée un classifieur binaire dont la tâche sera de distinguer la classe concernée de toutes les autres données (qui sont alors regroupées dans une classe "autre"). Lorsqu'on veut classifier une nouvelle donnée, on effectue une prédiction avec tous les classifieurs et on récupère la classe qui n'aura pas été classifiée comme étant "autre".
\end{description}

Les algorithmes de type \textit{one-vs-one} sont utilisés lorsqu'on possède des données ayant une prépondérance d'une classe sur une autre (par exemple, une classe qui serait représentée par 80\% des échantillons), au risque d'utiliser plus de ressources processeur.

\subsubsection{Ordre de grandeur des systèmes développés}

Jusqu'ici les systèmes développés sont adaptés à quelques dizaines ou une centaines d'utilisateurs mais guère plus. Cependant dans la vraie vie, il n'est pas rare que les systèmes d'authentification contrôlent l'accès à des services pour des millions d'utilisateurs tous différents. Le problème est que certains algorithmes de classifications présentent une baisse très importantes de la fiabilité avec l'augmentation du nombre d'utilisateurs \cite{panasiuk2016}. Il y a donc un problème de ces méthodes de classification car elles ne permettent pas de développer des systèmes d'authentification qui sont adaptés à différents ordres de grandeur. C'est la raison pour laquelle nous avons choisi d'orienter notre travail sur l'authentification d'une personne unique sur un ordinateur personnel. En effet, l'essentiel des travaux du domaine proposent des systèmes éprouvés sur un nombre restreint d'utilisateurs, nous pourrons donc nous appuyer sur ces travaux. Par ailleurs, il est beaucoup plus difficile d'évaluer les performances d'un système développé pour des millions d'utilisateurs, notamment parce qu'il n'existe pas de bases de données permettant de l'éprouver.

\subsection{Évaluation du système}

En biométrie, l'évaluation du système doit se faire avec beaucoup de données. C'est à cet effet qu'ont été créées les bases de donnée publiques tels que DSN-2009\cite{killourhy2009} et GREYC Keystroke\cite{giotGREYC}, à l'instar des bases de données déjà existantes concernant des données d'empreintes digitales ou d'iris dans les domaines de la biométrie morphologique. La composition de ces bases de données nous permet d'entraîner ou de tester un algorithme sur un utilisateur unique de la base de données avant d'utiliser le reste des données pour simuler des tentatives d'authentification par des imposteurs.

L'évaluation du système s'effectue en testant toutes les données contenues dans la base de données en authentification et on relève :

\begin{itemize}
  \item Le nombre de tentatives provenant d'un utilisateur légitime
  \item Le nombre de tentatives provenant d'un imposteur
  \item Le nombre de rejets lorsque l'utilisateur est légitime
  \item Le nombre de confirmations lorsque l'utilisateur est légitime
  \item Le nombre de rejets lorsque l'utilisateur est un imposteur
  \item Le nombre de confirmations lorsque l'utilisateur est un imposteur
\end{itemize}

Le FRR s'obtient en divisant le nombre de rejets lorsque l'utilisateur est légitime sur le nombre de tentatives provenant d'un utilisateur légitime. Il indique à quel point le système est restrictif (plus le FRR est bas, plus le système est restrictif).

Le FAR s'obtient en divisant le nombre de confirmations lorsque l'utilisateur est un imposteur sur le nombre de tentatives provenant d'un imposteur. Il indique à quel point le système est sécurisé (plus le FAR est bas, plus le système est sécurisé).

Ces deux métriques sont les plus significatives pour juger des performances d'un système d'authentification par biométrie et doivent se situer en dessous des seuils définis par la norme européenne EN-50133-1


\section{Conclusion}

La méthode suivie pour l'entraînement de la \textit{One-Class SVM} et son utilisation produit des résultats peu satisfaisants sur la base de données GREYC. En pratique, à partir des données que nous avions nous même recueillies et exploitées, nous obtenions des résultats plutôt satisfaisants. L'étude des données du GREYC nous montre qu'il y a un certain manque de finesse dans la capture des données et nous pensons que l'utilisation d'un méthode de capture plus précise permettrait d'améliorer les résultats obtenus précedemment.

% Bibliographie (faut voir plus tard pour que ça s'affiche dans le sommaire)

\bibliography{RapportBibliographique}

\end{document}
