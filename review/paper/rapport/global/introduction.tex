\section{Introduction}

Notre société moderne a été transformée par la démocratisation des outils informatiques. La transformation numérique du travail est un des défis principaux posés aux entreprises d'aujourd'hui, et la vie quotidienne de milliards d'individus dépend de l'accès à des services numériques en ligne, et les profils associés.

L'accès aux données confidentielles stockées dans des ressources informatiques est devenu tellement central que les enjeux de sécurité et de protection de la vie privée se retrouvent partout. Le RGPD, la dernière législation européenne sur la protection des données personnelles, atteste de l'urgence du sujet.

Pour protéger les données, les solutions qui ont toujours accompagné l'informatique sont systèmes permettant de \textbf{prouver son identité}. Le moyen le plus répandu et auquel le public s'est habitué est de retenir un mot de passe et/ou un nom d'utilisateur. Cependant, de nombreuses failles de sécurité ont montré que ces systèmes avaient leur limites. En effet, un mot de passe complexe à retenir ne garanti pas qu'il soit difficile à trouver par une machine. De plus, la complexité de ce mode d'authentification augmente avec l'accroissement du nombre de mots de passe différents à retenir. Ainsi, la plupart des failles de cette méthode d'authentification sont humains :

\begin{itemize}
	\item il est difficile de retenir autant de mots de passe que ce qu'il est nécessaire pour mener une vie quotidienne normale ;
	\item la solution la plus couramment adoptée est la réutilisation de mot de passe ;
	\item la répétition des mots de passe et la multiplication des endroits où ils sont stockés sont idéals pour les attaquants qui sont assistés de programmes beaucoup plus doués que les humains pour deviner des mots de passe et les retenir.
\end{itemize}

Par conséquent, la recherche et l'industrie ont tenté de créer différents moyens d'authentification qui seraient attaché à l'individus plutôt que d'être conditionnés à leur capacité à choisir des mots de passe complexe et à ne pas les réutiliser. Les \textit{keystroke dynamics} font partie de ce groupe de solutions.

Comme le champ de la recherche en la matière est fortement dépendant de concepts qui lui sont propre, nous commencerons par une partie de définitions. Nous examinerons ce qui vient de la biométrie elle-même, ce qui est propre aux \textit{keystroke dynamics} et ce qui est apporté par le \textit{machine learning}.

Par la suite, nous examinerons les enjeux d'acquisition des données, un point crucial pour la recherche en biométrie.

Enfin nous examinerons les différentes méthodes de classification utilisées, et les métriques qui permettent l'évaluation des modèles développés.
