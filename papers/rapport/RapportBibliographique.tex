% WELCOME WELCOME WELCOME

% Bienvenue dans notre rapport bibliographique en LaTeX

% Les paquets importés et les paramêtres de ceux-ci peuvent se trouver dans headers.tex

% La bibliographie peut se trouver dans RapportBibliographique.bib

% Les synthèses individuelles sont à mettre dans 'papers'
% Elles peuvent être inclus avec la commande \includesynthese{filename}
% Pas besoin d'indiquer l'extension, LaTeX va chercher automatiquement pour des *.tex

% Pour faire une citation, il faut utiliser la commande \cite{reference}
% La reference peut être trouvée dans le fichier de bibliographie

\documentclass[a4paper,11pt]{article}
\usepackage[T1]{fontenc}
\usepackage[utf8]{inputenc}
\usepackage{lmodern}
\usepackage[francais]{babel}
\usepackage{color}
\usepackage{hyperref}
\usepackage{cite} 

\hypersetup{
  colorlinks=true,
  linkcolor=blue,
  urlcolor=blue,
  citecolor=blue,
  linktoc=all
}

\newcommand\includesynthese[1]{\include{./papers/#1}}


% Ici on indique le titre et les auteurs (\\ pour passer une ligne dans la liste d'auteurs)

\title{Rapport bibliographique}
\author{Rémi Bourgeon\\
Franciszek Dobrowolski\\
Rodolphe Houdas\\
Timothee Jouglet}

% Let's start that shit

\begin{document}

% On ajoute la page de garde et la page de sommaire

\maketitle
\newpage
\tableofcontents
\newpage

% ABSTRACT

\begin{abstract}
Dans ce document, nous présenterons la dynamique de clavier et son utilisation dans le cadre de l'authentification par biométrie.
\end{abstract}

\newpage

% Le lexique se situe dans Lexique.tex

\section{Lexique}

\subsection{Biométrie}

\begin{description}
  \item[DDF (Dynamique De Frappe)] Rythme de frappe au clavier
  \item[DET (Detection Error Tradeoff)] Courbe traçant le taux de faux négatifs (FRR) en fonction du taux de faux positifs (FAR).
  \item[EER (Equal Error Rate)] Mesure utilisée en biométrie lorsque le FAR et le FRR sont égaux.
  \item[Enregistrement (\textit{enrollment})] Démarche à suivre par un utilisateur pour s'enregistrer dans un système biométrique
  \item[FAR (False Acceptance Rate)] Taux de faux positifs, c'est-à-dire quand l'algorithme accepte un imposteur.
  \item[FRR (False Rejectance Rate)] Taux de faux négatifs, c'est-à-dire quand l'algorithme refuse un individu légitime .
  \item[FTA (Failure To Acquire)] Échec de la capture. Dans le cas de la dynamique de frappe, il peut s'agir d'une mauvaise saisie.
  \item[FTE (Failure To Enroll)] Échec de l'enregistrement d'un nouvel utilisateur, principalement dû à un FTA.
  \item[\textit{Keystroke Dynamics}] champ de la biométrie comportementale qui utilise les données de frappe au clavier pour reconnaître un utilisateur.
  \item[ROC (Receiver Operating Characteristic)] Courbe traçant le taux de vrais positifs en fonction du taux de faux positifs.
\end{description}

\subsection{Machine Learning}

\begin{description}
  \item[kNN (k-Nearest Neighbors)] Algorithme de classification qui utilise la proximité avec k-voisins.
  \item[SVM (State Vector Machine)] Algorithme de classification, généralisation des classifieurs linéaires dont une des particularités est la transformation de l'espace de représentation lorsque les données ne sont pas linéairement séparables. 
\end{description}

\subsection{Informatique}

\begin{description}
  \item[Efficience] Optimisation de la consommation des ressources informatiques (RAM, processeur...) 
  \item[\textit{Timestamp}] Donnée temporelle permettant d'identifier une date avec une précision variant entre la nanoseconde et la milliseconde suivant le format.
\end{description}

% What are those ? On les as vus quelque part ?

% PNN ( Probabilistic Neural Network ) : un type du réseau neuronal dans la reconnaissance des patterns
% ADL (en. LDA) ( Analyse discriminante linéaire ) : expliquer et de prédire l’appartenance d’un individu à une classe
    


% On ajoute les synthèses individuelles des publications

\includesynthese{Hu}
\includesynthese{Buchoux}
\includesynthese{GiotSVM}

% Bibliographie (faut voir plus tard pour que ça s'affiche dans le sommaire

\bibliography{RapportBibliographique}
\bibliographystyle{ieeetr}

\end{document}
