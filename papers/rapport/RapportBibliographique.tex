% WELCOME WELCOME WELCOME

% Bienvenue dans notre rapport bibliographique en LaTeX

% Les paquets importés et les paramêtres de ceux-ci peuvent se trouver dans headers.tex

% La bibliographie peut se trouver dans RapportBibliographique.bib

% Les synthèses individuelles sont à mettre dans 'papers'
% Elles peuvent être inclus avec la commande \includesynthese{filename}
% Pas besoin d'indiquer l'extension, LaTeX va chercher automatiquement pour des *.tex

% Pour faire une citation, il faut utiliser la commande \cite{reference}
% La reference peut être trouvée dans le fichier de bibliographie

\documentclass[a4paper,11pt]{article}
\usepackage[T1]{fontenc}
\usepackage[utf8]{inputenc}
\usepackage{lmodern}
\usepackage[francais]{babel}
\usepackage{color}
\usepackage{hyperref}
\usepackage{cite} 

\hypersetup{
  colorlinks=true,
  linkcolor=blue,
  urlcolor=blue,
  citecolor=blue,
  linktoc=all
}

\newcommand\includesynthese[1]{\include{./papers/#1}}


% Ici on indique le titre et les auteurs (\\ pour passer une ligne dans la liste d'auteurs)

\title{Rapport bibliographique}
\author{Rémi Bourgeon\\
Franciszek Dobrowolski\\
Rodolphe Houdas\\
Timothee Jouglet}

% Let's start that shit

\begin{document}

% On ajoute la page de garde et la page de sommaire

\maketitle
\newpage
\tableofcontents
\newpage

% ABSTRACT

\begin{abstract}
Dans ce document, nous présenterons la dynamique de clavier et son utilisation dans le cadre de l'authentification par biométrie.
\end{abstract}

\newpage

% Le lexique se situe dans Lexique.tex

\section{Lexique}

\subsection{Biométrie}

\begin{description}
  \item[DDF (Dynamique De Frappe)] Rythme de frappe au clavier
  \item[DET (Detection Error Tradeoff)] Courbe traçant le taux de faux négatifs (FRR) en fonction du taux de faux positifs (FAR).
  \item[EER (Equal Error Rate)] Mesure utilisée en biométrie lorsque le FAR et le FRR sont égaux.
  \item[Enregistrement (\textit{enrollment})] Démarche à suivre par un utilisateur pour s'enregistrer dans un système biométrique
  \item[FAR (False Acceptance Rate)] Taux de faux positifs, c'est-à-dire quand l'algorithme accepte un imposteur.
  \item[FRR (False Rejectance Rate)] Taux de faux négatifs, c'est-à-dire quand l'algorithme refuse un individu légitime .
  \item[FTA (Failure To Acquire)] Échec de la capture. Dans le cas de la dynamique de frappe, il peut s'agir d'une mauvaise saisie.
  \item[FTE (Failure To Enroll)] Échec de l'enregistrement d'un nouvel utilisateur, principalement dû à un FTA.
  \item[\textit{Keystroke Dynamics}] champ de la biométrie comportementale qui utilise les données de frappe au clavier pour reconnaître un utilisateur.
  \item[ROC (Receiver Operating Characteristic)] Courbe traçant le taux de vrais positifs en fonction du taux de faux positifs.
\end{description}

\subsection{Machine Learning}

\begin{description}
  \item[kNN (k-Nearest Neighbors)] Algorithme de classification qui utilise la proximité avec k-voisins.
  \item[SVM (State Vector Machine)] Algorithme de classification, généralisation des classifieurs linéaires dont une des particularités est la transformation de l'espace de représentation lorsque les données ne sont pas linéairement séparables. 
\end{description}

\subsection{Informatique}

\begin{description}
  \item[Efficience] Optimisation de la consommation des ressources informatiques (RAM, processeur...) 
  \item[\textit{Timestamp}] Donnée temporelle permettant d'identifier une date avec une précision variant entre la nanoseconde et la milliseconde suivant le format.
\end{description}

% What are those ? On les as vus quelque part ?

% PNN ( Probabilistic Neural Network ) : un type du réseau neuronal dans la reconnaissance des patterns
% ADL (en. LDA) ( Analyse discriminante linéaire ) : expliquer et de prédire l’appartenance d’un individu à une classe
    


% On ajoute les synthèses individuelles des publications

\includesynthese{Monrose} %1997
\includesynthese{Buchoux} %2008
\includesynthese{Hu} %2008
\includesynthese{Rosenberger} %2009
\includesynthese{Giot_Benchmark} %2009
\includesynthese{GiotSVM} %2009
\includesynthese{Giot_Web} %2012

% Synthèse globale

\section{Introduction}
\section{Définitions}
Champ de la recherche fortement associé à la biométrie, dont il est une branche, et au \textit{Machine Learning}, qui est un des outils les plus utilisés
\subsection{Biométrie}

Les auteurs de \bibentry{Buchoux2008} soulignent qu'il existe trois approches pour identifier une personne :

\begin{itemize}
	\item utiliser ce que l'utilisateur \textbf{sait}, donc l'approche du mot de passe ;
	\item utiliser ce que l'utilisateur \textbf{a}, c'est l'approche retenue par les clés physiques et les systèmes double authentification par SMS ;
	\item utiliser ce que l'utilisateur \textbf{est}, c'est l'approche de la biométrie.
\end{itemize}

Un système de contrôle biométrique est un système automatique de mesure basé sur la reconnaissance de caractéristiques propres à l'individu. Il s'agit de l'authentification basée sur les caractéristiques physiques ou comportementales. C'est à dire ce que l'individu est ou ce qu'il sait faire.

\subsubsection{L'analyse morphologique}

L'analyse des caractéristiques physiques est aussi appelée l'analyse morphologique et peut se pratiquer avec les empreintes digitales, l'iris, la morphologie de la main, ainsi qu'avec les traits du visage.

L'analyse des empreintes digitales consiste à scanner le dessin formé par les lignes de la peau des doigts d'un individu. Les empreintes ont l'avantage d'être uniques et pratiquement immuables au cours de la vie.

Pour scanner l'iris une caméra acquiert son dessin et le compare à un fichier d'identification contenant le modèle de référence stocké pour un individu.

La morphologie de la main est particulièrement populaire aux États-Unis. Elle consiste à mesurer des caractéristiques de la main comme la longueur et la largeur des doigts ou la forme des articulations. Le désavantage de cette méthode est la similitude des résultats pour les personnes venant de la même famille, ayant une morphologie semblable.

L'analyse du visage se base sur la mesure des distances entre les éléments stratégiques du visage. On analyse l'écartement des yeux,  les arêtes du nez, les commissures des lèvres, les oreilles et le menton. Les systèmes de reconnaissance du visage sont actuellement en évolution et permettent de prendre des échantillons en mouvement ou prendre en considération le vieillissement d'un individu.


\subsubsection{L'analyse comportementale}

L'analyse comportementale se concentre sur ce que l'individu sait faire de manière unique. Celle-ci peut reposer sur l'analyse de la voix, la dynamique des signatures ou la dynamique de frappes au clavier.

L'analyse de la voix consiste à la mesure des intonations qui sont ensuite comparées à l'empreinte pour confirmer l'identité. Ce système est utilisé pour accéder à certains service bancaires ou des lignes de service après vente automatisées.

La dynamique des signatures se concentre sur les gestes effectues lors de la signature. Cette analyse considère la direction et la pression d'un tracé du stylo et combine ces informations à la forme de la signature afin de déterminer l'identité.

La dynamique de frappe au clavier repose sur un principe similaire, mais au lieu d'observer les gestes, on mesure les temps de pression et de relâchement des différentes touches.

\subsubsection{Analyse des performances}

Pour évaluer l'efficacité des différents systèmes biométriques nous pouvons considérer trois facteurs majeur:

\begin{itemize}
\item la performance - analyse des données statistiques sur le performance du système, comme EER, FTE, FTA ou temps de traitement
\item l'acceptabilité - les informations sur la perception et l'acceptation par l'utilisateur 
\item sécurité - quantifie la sécurité du système, par example combien de fraudes un imposteur peut utiliser
\end{itemize}

Ces trois caractéristiques doivent être prises en compte simultanément pour juger sur la performance d'un système biométrique. Un niveau bas de fausses identifications ne pourra pas être significatif si le taux de refus des utilisateurs légitimes est élevé. Dans se cas la solution sera condamné à ne pas être utilisée.

\subsubsection{Conclusion}

Les méthodes d'authentification biométriques présentent des avantages de fiabilité, rapidité et facilité d'utilisation. Certains domaines peuvent encore être approfondis, notamment la dynamique de la frappe au clavier.





\subsection{Dynamique de frappe}
\label{subsec:ddf}

Les \textit{keystroke dynamics}, ou en français la \textbf{dynamique de frappe au clavier} est une méthode biométrique comportementale. Depuis le début de la recherche dans ce domaine\cite{rand}, l'intérêt de cette méthode a été de proposer une identification peu coûteuse en matériel et en traitement \cite{monrose1997}. La donnée en la matière est théoriquement abondante, et l'acquisition des données est transparente pour les utilisateurs.\\

La littérature distingue deux types de systèmes biométriques utilisant la dynamique de frappe :\\


\begin{description}
  \item[Statique] Il s'agit de l'identification ou de l'authentification d'un utilisateur à l'aide d'un mot de passe fixe choisi par l'utilisateur (ou l'expérimentateur dans le cas des recherches). Cela permet par exemple de faire de l'authentification bi-factorielle combinant un mot de passe et la dynamique de frappe.
  \item[Dynamique] Il s'agit de l'identification ou de l'authentification d'un utilisateur sur du texte libre. Cela permet par exemple de faire du \textit{monitoring} de l'identité de l'utilisateur et de verrouiller le système lorsqu'un utilisateur non autorisé est détecté.\\
\end{description}

D'après Romain Giot \cite{giotBenchmark}, un système de \textit{keystroke dynamics} est toujours composé de deux modules :\\

\begin{itemize}
	\item Un module d'enregistrement (\textit{enrollment}) des utilisateurs ;
	\item Un module de vérification.\\
\end{itemize}

Le module d'enregistrement est ce qui permet de capter la donnée de frappe au clavier. Le logiciel développé à l'intention va capter les \textbf{événements claviers} émis par la frappe de l'utilisateur.

En ce qui concerne le module de vérification, il se base en général sur un \textbf{classificateur}, qui peut faire partie de plusieurs catégories : 

\begin{itemize}
	\item une seule classe : détection des imposteurs \bibentry{killourhy2009};
	\item une classe par utilisateur : identification de l'utilisateur parmi plusieurs \bibentry{monrose1997}.
\end{itemize}



\subsection{\textit{Machine Learning}}

Le \textit{machine learning} est un ensemble d'algorithme d'analyses statistiques qui ont la particularité d'améliorer leurs performances à résoudre une tâche sans avoir été programmés pour résoudre cette tâche en particulier.\\

Au plus haut niveau, les algorithmes de \textit{machine learning} peuvent être divisés en deux types : \textbf{supervisés} et \textbf{non-supervisés}. D'autres types tels que les algorithmes semi-supervisés existent mais ne sont pas couverts par ce document car réservé à des emplois bien spécifiques.\\

Les algorithmes supervisés peuvent eux-même être divisés en algorithmes de \textbf{classification} et en algorithmes de \textbf{régression}. Nous nous intéresserons ici qu'aux algorithmes de classification qui consistent à associer une classe à une donnée inconnue (par exemple reconnaître un objet dans une image). À ce titre, le \textit{machine learning} est particulièrement adapté pour des tâches d'authentification par classification.\\

Les algorithmes supervisés prennent en entrée des échantillons dont on connaît la classe et à laquelle on souhaite appliquer une frontière de décision qui permettra de prédire la classe associée à une nouvelle donnée. Dans le cadre d'une authentification, l'algorithme de \textit{machine learning} aura comme classes les utilisateurs (une classe par utilisateur) et comme attributs les données mesurées (il peut s'agir des temps entre les pressions sur les touches\cite{giotGREYC} ou des temps de frappe des n-graphes\cite{Hu2008,bergadano2002,gunetti2005}).\\

Les algorithmes non-supervisés prennent en entrée des échantillons auxquels on n'attache aucune valeur autre que les attributs, que l'on cherche à constituer en groupes afin d'en déduire une valeur commune et pour pouvoir prédire quelle valeur sera rattaché à une nouvelle donnée. Un exemple typique est la recherche de visages semblables dans une collection de photos où les personnes ne sont pas identifiées, afin d'identifier de manière unique les différentes personnes présentes dans ces photos.\\

Les algorithmes de \textit{machine learning} supervisés qui font de la classification multi-classe (c'est-à-dire une classification entre trois classes ou plus) sont divisés en deux types :\\

%http://scikit-learn.org/stable/modules/svm.html#multi-class-classification

\begin{description}
  \item[\textit{one-vs-one}] : pour chaque couple de classe, on créé un classifieur binaire. Pour classifier une nouvelle donnée, on effectue une prédiction avec tous les classifieurs et on récupère la classe ayant été la plus prédite parmis tous les classifieurs.
  \item[\textit{one-vs-rest}] : pour chaque classe, on créé un classifieur binaire dont la tâche sera de distinguer la classe concernée de toutes les autres données (qui sont alors regroupées dans une classe "autre"). Lorsqu'on veut classifier une nouvelle donnée, on effectue une prédiction avec tous les classifieurs et on récupère la classe qui n'aura pas été classifiée comme étant "autre".\\
\end{description}

Les algorithmes de type \textit{one-vs-one} sont utilisés lorsqu'on possède des données ayant une prépondérance d'une classe sur une autre (par exemple, une classe qui serait représentée par 80\% des échantillons), au risque d'utiliser plus de ressources processeur.\\

Parmis les publications que nous avons étudié, nous avons pu lire que dans la majorité des cas, les algorithmes à base de \textit{machine learning} sont plus performantes que les méthodes statistiques, autant sur le plan des ressources processeurs\cite{Hu2008} que sur le plan de la fiabilité en termes biométriques\cite{giotBenchmark}. Parmis les algorithmes à base de \textit{machine learning} aperçus dans la littérature, nous pouvons citer :\\

\begin{description}
  \item[kNN\cite{Hu2008} (\textit{k-Nearest Neighbors})]
  \item[SVM\cite{giotSVM} (\textit{State Vector Machine})] : il s'agit d'un algorithme de classification qui généralisation des classifieurs linéaires dont les particularités sont la transformation de l'espace de représentation lorsque les données ne sont pas linéairement séparables et la maximisation de la marge au niveau des échantillons les plus proches de la frontière de décision afin de redéfinir une frontière de décision plus précise.
  \item[ANN (\textit{Artificial Neural Network})] : il s'agit d'un algorithme comprenant une couche de neurones d'entrée, une couche de neurones de sortie et une ou plusieurs couches de neurones intermédiaires. Les neurones intermédiaires agissent comme des coefficients qui sont appliqués dans une équation dont les paramètres d'entrée sont les neurones d'entrée et la sortie, la classe correspondante aux données entrée. Sa construction se révèle très simple et permet de résoudre des problèmes très complexes. Son efficacité sur des jeux de données comprenant un grand nombre d'attributs ( 50) en fait un outil de choix pour la discrimination de schémas complexes tels que des enregistrements de frappe au clavier. Il a comme désavantage d'être très gourmand en mémoire et en ressources processeurs, tant pour l'entraînement que pour la prédiction.
  \item[One-class SVM\cite{oneclassSVM}] : il s'agit d'une SVM non-supervisée. Elle est utilisée pour des tâches d'\textit{outlier detection}, c'est-à-dire la détection de données anormales, n'appartenant pas au groupe de données initial. Pour être utilisée, elle ne nécessite qu'un groupe de données appartenant à une seule classe. Elle est donc parfaitement adaptée à la détection d'un seul utilisateur en absence de données d'imposteurs.
\end{description}


%En définitive les méthodes d'identification orbitent autour de deux catégories :

%* Méthodologie statistique: pour tout échantillon à évaluer, on calcule des
%caractéristiques statistiques de la donnée d'entrée et on parcourt toute la base
%d'échantillons enregistrés pour trouver celui qui s'en rapproche le plus ;
%* Utilisation d'algorithmes de Machine Learning : pour tout échantillon à
%évaluer, on le traite pour le transformer en vecteur et le modèle va le
%rapprocher de la classe correspondante.

%Attention : les deux méthodes ne sont pas exclusives. Après tout les algorithmes
%de classification en Machine Learning sont d'abord et avant tout des méthodes
%statistiques automatisées. De plus, on peut utiliser des échantillons traités au
%moyen de la statistique pour "entraîner" son modèle de Machine Learning.
%[Hu, Ginrich]

\section{Acquisition}
L'état de l'art du domaine nous permet aussi d'établir qu'il existe des métriques pour évaluer le résultat de l'acquistion des données \cite{giotWeb}.\\
% à développer pour faire une transition
Les différentes catégories d'enjeux liés à la discipline sont associés à ces deux pôles. Nous allons d'abord examiner quels sont les difficultés à anticiper en matière d'acquisition des données.

\subsection{Protocole d'acquisition}

Romain Giot \textit{et al.} \cite{giotGREYC} soulignent la difficulté de constituer une base de données conséquente pour la recherche en \textit{keystroke dynamics}. En effet, de nombreux paramètres sont susceptibles d'ajouter du bruit dans la donnée captée, voire de la rendre inutilisables. Cependant la littérature n'est pas unanime quant au fait que des erreurs de saisie viendraient renforcer la fiabilité des modèles ou la diminuer.

Romain Giot \textit{et al.} \cite{giotGREYC} ainsi que Killourhy et Maxion \cite{killourhy2009} sont ont donc tenté de répondre à cette problématique en constituant des bases de données conséquentes, basées sur un protocole d'acquisition et un formatage documentés.

Ces travaux nous permettent de nous insérer dans un ensemble de publications réutilisant ces données pour rendre les résultats comparables, ce qui n'était pas vraiment possible avant 2009.

Cependant, il est tout de même utile de souligner qu'il reste encore beaucoup à faire en la matière, notamment en comparaison des autres champs d'étude du \textit{Machine Learning}, qui disposent de bases de données de millions d'images pour la reconnaissance des visages, par exemple.

De plus, et afin de mettre en place un protocole d'acquisition de la donnée dans une application concrète, il faut avoir à l'esprit les enjeux soulignés tout au long de \bibentry{giotBenchmark}

L'état de l'art du domaine nous permet aussi d'établir qu'il existe des métriques pour évaluer le résultat de l'acquistion des données \cite{giotWeb}.

Afin d'avoir un système "\textit{Keystroke Dynamics}" efficace et fiable (c'est à dire laissant entrer les bon utilisateurs lors de l'analyse et restreignant l'accès à des imposteurs), il est indispensable de traiter et filtrer la donnée en amont. Il est également nécessaire d'établir des critères et de filtres afin de  limiter les erreurs. Nous allons donc ici nous intéresser aux critères a regarder afin d'obtenir une donnée pertinente.

\subsubsection{Qualité de la donnée}
Un des débats revenant le plus souvent est l'acceptation de l'erreur lors de l'acquisition de la donnée. En effet, selon le document de Romain Giot sur les benchmark (\cite{giotBenchmark}), la correction d'une erreur change la manière de taper un texte, vu qu'il faut  prendre en compte les \textit{inputs} nécessaire pour rectifier le texte erroné. Par conséquent, la majorité des base de données du domaine ne permettent pas l'erreur lors de l'acquisition: l'utilisateur se voit donc forcer de reprendre l'acquisition de zéro en cas d'erreur. Néanmoins, cette particularité reste un champ de recherche possible pour l'amélioration de la technologie.

Une autre question récurrente dans le domaine des \textit{Keystroke Dynamics} concerne les mots de passes : doit-on forcement utiliser un unique mot de passe/une unique paraphrase pour tous les utilisateurs de la phase d'acquisition. Dans la majorité des base de données, une paraphrase ou un mot de passe unique est pré défini et utilisée par les utilisateurs. Cette méthode a l'avantage d'être peu couteux en terme de temps et moins complexe à gérer au niveau des bases de données (l'information de réussite dans la frappe de cette paraphrase est une valeur booléenne). Néanmoins cette pratique reste peu réaliste par rapport à l'utilisation que l'on voudrait en faire (permettre une double authentification, afin d'accroitre la sécurité lors d'une connexion). De plus, l'utilisateur doit donc apprendre une chaine de caractères ce qui prend du temps et entraine une évolution de la frappe au fil des acquisitions.
Des tests ont cependant été effectués par le laboratoire du Greyc (\cite{giotWeb}) par le biais d'une application web (Un contexte plus réaliste à la mise en pratique de cette technologie). Il s'avère par le biais de ces expériences comparatives, qu'il n'y ait pas de différence notable de fiabilité et de performance entre un mot de passe unique et un mot de passe choisi par l'utilisateur.
En conclusion de cela, il n'est pas nécessaire d'utiliser un mot de passe unique pour les tests, mais reste fortement recommandé dans une souci de simplicité.

Les données de tests doivent néanmoins se plier à certains critères précis :

\begin{itemize}
	\item L'entropie de la donnée
	\item La taille de la donnée
	\item La complexité de la donnée
\end{itemize}

\subsection{Critères propre à l'utilisateur et à l'environnement d'acquisition}



\subsection{Importance des bases de données publiques}

\subsection{Ingénierie des caractéristiques}

\section{Classification/Évaluation}
Une fois l'acquisition des données utilisateur réalisée, il reste à produire un système de vérification capable de discriminer un imposteur par rapport à un ou plusieurs utilisateurs authentiques. Les algorithmes et méthodologies utilisées varient en fonction des objectifs recherchés.

\subsection{Méthodologies de classification}
La première étape est de construire un modèle de vérification en classant les données disponibles en plusieurs "classe" de profils. Plusieurs méthodes possibles existent pour classer les données :
\begin{itemize}
	\item
	Méthode statistique ;
	\item
	Algorithmes de \textit{Machine Learning} \bibentry{Hu2008};
	\item
	Réseaux de neurones. À proprement parler il s'agit aussi de \textit{Machine Learning} mais la littérature a tendance à les séparer du reste car ils ont des propriétés propres et requièrent des performances bien plus importantes que les algorithmes classiques.
\end{itemize}

Les méthodes utilisées varient aussi en fonction de l'objectif que l'on veut faire remplir au modèle de vérification :

\begin{itemize}
	\item une seule classe : détection des imposteurs \bibentry{killourhy2009};
	\item ne classe par utilisateur : identification de l'utilisateur parmi plusieurs \bibentry{monrose1997}.
\end{itemize}

\subsection{Métriques}
FAR / FRR

\subsection{Ordre de grandeur des systèmes développés}

Jusqu'ici les systèmes développés sont adaptés à quelques dizaines ou une centaines d'utilisateurs mais guère plus. Cependant dans la vraie vie, il n'est pas rare que les systèmes d'authentification contrôlent l'accès à des services pour des millions d'utilisateurs tous différents. Le problème est que certains algorithmes de classifications présentent une baisse très importantes de la fiabilité avec l'augmentation du nombre d'utilisateurs \cite{panasiuk2016}. Il y a donc un problème de ces méthodes de classification car elles ne permettent pas de développer des systèmes d'authentification qui sont adaptés à différents ordres de grandeur. C'est la raison pour laquelle nous avons choisi d'orienter notre travail sur l'authentification d'une personne unique sur un ordinateur personnel. En effet, l'essentiel des travaux du domaine proposent des systèmes éprouvés sur un nombre restreint d'utilisateurs, nous pourrons donc nous appuyer sur ces travaux. Par ailleurs, il est beaucoup plus difficile d'évaluer les performances d'un système développé pour des millions d'utilisateurs, notamment parce qu'il n'existe pas de bases de données permettant de l'éprouver.

% Bibliographie (faut voir plus tard pour que ça s'affiche dans le sommaire)

\bibliography{RapportBibliographique}
\bibliographystyle{ieeetr}

\end{document}
