\documentclass[a4paper,11pt]{article}
\usepackage[utf8]{inputenc}
\usepackage{lmodern}
\usepackage[francais]{babel}
\usepackage{color}
\usepackage{hyperref}
\usepackage{cite} 

\hypersetup{
    colorlinks=true, % make the links colored
    linkcolor=blue, % color TOC links in blue
    urlcolor=blue, % color URLs in red
    linktoc=all % 'all' will create links for everything in the TOC
}

\newcommand\includesynthese[1]{\include{./papers/#1}}

\begin{document}

\title{A Review on the Public Benchmark Databases for Static Keystroke Dynamics\cite{giotBenchmark}}
\maketitle

\section{Abstract}

Le document nous donne un point de vue comparatif des performances de différentes bases de données "\textit{Keystroke Dynamics}".    
L'étude nous permet également d'avoir un aperçu sur le vocabulaire trouvable dans la littérature et des divers critères de comparaisons et de performances d'une base de données.
Enfin, le document dresse une étude comparative (accompagné de divers graphes) des divers bases de données citées.

\section{Introduction}

L'auteur commence en nous définissant le terme de "\textit{Keystroke Dynamics}" et en affirmant l'interet de ce domaine dans l'identification et l'authentification de personnes.Par la suite, nous trouvons une explication détaillé sur la composition d'un système de dynamique de frappe (KDS), celui-ci divisé en deux parties : la phase d'enregistrement et la phase de vérification.

\section{Benchmark des bases de données publiques existantes en \textit{Keystroke Dynamics}}

Les différentes bases de données traitant de dynamique de frappe ("\textit{Keystroke dynamics}"ou  KD), nous sont présentés, avec des caractéristiques sur la densité de données (nombre de sessions de captures, nombre de captures par session ...) et sur les méthodes d'acquisition (durée inter-capture, inter-session, ...).  Les bases de données présentes dans le document, sont les plus communes dans le domaine des \textit{Keystroke Dynamics} (exemple : la Base "Greyc")

\section{Caractérisation d'une base de données en \textit{Keystroke Dynamics}}

Un état de l'art des caractéristiques et des configurations de captures est ensuite effectué, afin de comprendre la manière dont sont effectués les études.
Nous apprenons donc les détails techniques globaux, liés à l'acquisition de données (exemple : Durée d'acquisition des bases), aux personnes testées, et aux caractéristiques des mots de passes.
D'autres critères de performances sont  définis tel que "l'unicité", "l'inconsistance" ou encore la \textit{Discrimability}, afin de comprendre l'évaluation des bases de données

\section{Étude comparative des bases de données \textit{Keystroke Dynamics}}

Cette partie du document fait donc un comparatif globale des performances des diverses bases analysés, en mettant en conflit les divers paramètres présentés lors des précédente parties. Le texte est agrémenté de plusieurs tables de valeurs résumant au mieux les résultats.
Il y a également une comparaison sur les critères d'acquisitions en eux-même, permettant de mettre en évidence, certaines pratiques communes (exemple : Pas le droit à l'erreur lors de la frappe d'un mot de passe, durant la phase d'enregistrement)

\section{Pertinence du Document}

Cette étude comparative nous permet tout d'abord de comprendre les divers notions et autres vocabulaires ressortant souvent dans les différentes thèses.    
De plus, avec les divers figures de comparaisons des bases et les données présentes, nous avons une représentation des données et des bases de données, pouvant être trouvés lors des recherches.   
Enfin, le document nous dresse une liste de critères à la réalisation d'une base. Ces critères nous seront donc utile lors de la phase d'expérience.
\end{document}