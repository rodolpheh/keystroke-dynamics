\section{Deployment of Keystroke Analysis on a Smartphone\cite{buchoux2008}}

Attention : ce document ayant été écrit en 2008, les performances ont grandement évoluées depuis. L'efficacité des méthodes mises en places dans le document restent valables, même d'après les standards d'aujourd'hui.

\subsection{Introduction}

Les auteurs partent du postulat qu'avec le développement de la technologie du mobile, celui-ci est devenu un point d'entrée vers de plus en plus de données sensibles (personnellement et financièrement). À cet effet, ils prévoient une augmentation de la complexité des systèmes d'authentification pour mobiles afin de répondre à un risque de plus en plus grand.\\

Il existe, selon Wood, trois approches à l'authentification :\\

\begin{itemize}
  \item Utiliser ce que l'utilisateur sait
  \item Utiliser ce que l'utilisateur a
  \item Utiliser ce que l'utilisateur est\\
\end{itemize}

Cette dernière approche est la biométrie.\\

Selon les auteurs, une meilleure approche de la sécurité serait une sécurité multi-factorielle. Dans ce document, l'approche utilisée sera une approche bi-factorielle, utilisant un mot de passe et les données de frappe au clavier. Cette approche sera implémentée sur un téléphone portable.

\subsection{Revue de la littérature}

Les auteurs font une revue de la littérature sur la frappe au clavier et liste les scores de FAR et FRR obtenus en fonction de la technique de classification, des données mesurées, du nombre de sujet et de la méthode utilisée (statique ou dynamique). La revue de ces documents leur permet de conclure qu'il est conseillé d'utiliser les données de latence et de hold-time des touches, sauf pour les mobiles, pour lesquels seul la latence importe (Karatzouni et al. (2007)).

\subsection{Méthodologie}

Le matériel utilisé est, en termes de performances et d'après les standards actuels, plutôt faible. Nous n'allons donc pas décrire l'ensemble de la méthodologie qui concerne surtout le matériel et le logiciel utilisé. Nous retiendrons que les classificateurs utilisés sont des algorithmes basés sur la distance Euclidienne, la distance de Mahalanobis et un réseau de neurones de type Feed-Forward Multi-Layered Perceptron (autrement dit, le type de réseau de neurones le plus basique et le plus courant).\\

L'expérience commence par la prise de 20 mesures par utilisateur, pour les données d'entraînement, et de 10 mesures supplémentaires pour les données de test.

\subsection{Résultats}

Les auteurs observent un rapport de 10 entre les temps d'enregistrements des différents algorithmes (2 secondes pour la distance de Mahalanobis contre 210 secondes pour un réseau de neurones simples). Le temps de vérification est quasi-instantané avec les distances Euclidienne et de Mahalanobis. Les FAR et FRR sont biens meilleurs lorsqu'on utilise tout le clavier alphanumérique plutôt que juste le clavier numérique. La différence entre les méthodes par distance Euclidienne et distance de Mahalanobis n'est pas assez significative pour en tirer une conclusion. Pour des raisons de performances, le réseau de neurones n'a pas été évalué.

\subsection{Conclusion}

Les auteurs concluent qu'il est possible d'implémenter une authentification combinant mot de passe et frappe au clavier, à condition d'utiliser tous les caractères (pas d'approche par code PIN). Aujourd'hui, les performances des téléphones pourraient permettre de refaire l'étude et de la compléter avec des résultats sur l'utilisation d'un réseau de neurones.\\

Cette étude nous intéresse dans notre cas parce qu'elle précise les données significatives (temps de latence et hold-time) et montre qu'il est possible d'implémenter un algorithme d'authentification par dynamique de frappe sur une plateforme mobile, qui est un des sujets que nous souhaitions étudier.
