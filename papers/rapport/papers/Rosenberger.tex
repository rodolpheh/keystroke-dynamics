\section{2009 - GREYC Keystroke : a Benchmark for Keystroke Dynamics Biometric Systems}
Romain Giot
Mohamad El-Abed
Christophe Rosenberger

\subsection{Contexte}
Publication parue 29 ans après la première traitant de Keystroke Dynamics. Selon l'état de l'art présenté par les auteurs, celle-ci est la première à réaliser une aussi grande collecte ayant en vue l'universalité des données.

\subsection{Introduction}
Publication parue 29 ans après la première traitant de Keystroke Dynamics. Selon l'état de l'art présenté par les auteurs, celle-ci est la première à réaliser une aussi grande collecte ayant en vue l'universalité des données.

\subsection{Methode experimentale}
Les auteurs se sont engagés à construire une source de données pour les futures recherches. Afin de garantir la qualité des données, la collecte a été réalisée selon les critères pertinents relevés par les études préalables.

Les auteurs ont réalisés un logiciel permettant de saisir des échantillons de façon reproductible. Il s'agit des échantillons de type statique, tels que les mots de passe.

Pour construire le lot, 133 personnes ont participé en laissant de 5 à 107 échantillons. En 4 mois 7555 captures ont été réalisées.

Lors des essais les sujets ont été amenés à changer de matériel et espacer les saisies dans le temps.

Les sujets participants étaient en majeure partie masculins âgés de 18 à 25 ans.

\subsection{Résultat}
Ce travail se concentre sur la construction des matériaux pour les futures analyses.

Les données réunies sont disponibles sous forme brute et traité. Le logiciel permet d'enrichir le lot à condition de partager des résultats.

Comme il s'agit d'une analyse statique des mots de passe, on peut parler d'essai correct ou erroné. Les auteurs ont identifiés 6 facteurs causant des erreurs lors de la saisie.

\subsection{Conclusion vis à vis de notre travail}
Il est conseillé de se baser sur le lot de données existant plutôt que de construire un nouveau. Le choix des méthodes est argumenté et les conditions de collecte sont documentés. De cette façon les résultats obtenus pourront être comparés avec d'autres études.

\subsection{Problématique}
Pb1: Quelle influence peut-on avoir sur les algorithmes de l'analyse, en mélangeant pour un même utilisateur, des saisies venant des différents périphériques ?

Pb2: Peut-on exploiter le taux d'erreur de saisie d'un mot de passe comme facteur pour l'identification biométrique ?
