\section{Comparing Anomaly-Detection Algorithms for Keystroke Dynamics\cite{peacock2004}}

\bibentry{peacock2004}

\subsection{Résumé}
Depuis 25 ans, les chercheurs ont développé des systèmes d'authentification basés sur les  \textit{keystroke dynamics} en voulant renforcer les systèmes basés sur nom d'utilisateur et mot de passe. L'intention est de se baser sur des caractéristiques comportementales de l'utilisateur, réputées non imitables, pour venir renforcer le système d'authentification par mot de passe, sans en diminuer sa commodité d'utilisation.

Cependant le champ de la recherche en \textit{keystroke dynamics} a encore à surmonter de nombreux défis pour que cela devienne une méthode d'authentification biométrique plus utilisée.

Les auteurs de cette publication s'attachent à répertorier les tendances associées aux recherches en \textit{keystroke dynamics}, à la fois pour en souligner certain problèmes que pour proposer des solutions.

\subsection{Applications}
Les applications des \textit{keystroke dynamics} gravitent autour de :

\begin{itemize}
	\item \textbf{l'authentification}, soit l'utilisation des *keystroke dynamics* comme alternative ou comme renfort au mot de passe ;
	\item \textbf{l'identification}, soit la distinction d'un utilisateur parmi plusieurs connus par le système, au moyen des informations de frappe au clavier. Cette catégorie pouvant aussi regrouper la \textbf{supervision}, soit la détection d'un changement d'utilisateur au cours d'une session, ou l'altération du comportement d'un utilisateur.
\end{itemize}

\subsection{Évaluation}
Les principaux problèmes de la recherche sont associés aux moyens d'évaluation des résultats des différentes publications. Les métriques utilisées pour évaluer les méthodes d'authentification biométriques sont assez peu maîtrisées par les chercheurs en \textit{keystroke dynamics} et les auteurs de l'article donnent des références afin de consolider l'évaluation des performances des différents systèmes développés.

De plus, le manque de base de données commune à ce champ de la recherche rend la comparaison et la reproduction des résultats très limitée. La constitution des échantillons est un poids à porter par les chercheurs et implique une logistique et un protocole complexe, ce qui limite le temps dédié à améliorer la fiabilité du système d'authentification lui-même. Les auteurs soulignent notamment que la plupart des jeux de données constitués par les chercheurs se limitent à environ 25 utilisateurs, tandis que la réalité d'un service web est de fournir un accès à des millions d'utilisateurs.

\subsection{Frein au développement de la recherche}
Enfin, une autre catégorie de problèmes plus difficiles à évaluer regroupent :

\begin{itemize}
	\item les brevet et la propriété intellectuelle attachée à certaines formes de dispositifs d'authentification par \textit{keystroke dynamics} qui empêchent les chercheurs de diffuser leurs publications, leurs données et leur code source ;
	\item l'acceptation par le public de dispositifs enregistrant le comportement ;
	\item la protection et la confidentialité des données de frappe au clavier des utilisateurs.
\end{itemize}

\subsection{Apports de la publication}
Cette publication nous apporte un tour d'horizon des challenges posés par les \textit {keystroke dynamics} et qui attendaient encore résolution en 2004. Aujourd'hui,on sait que des chercheurs ont essayé de développer des bases de données pour répondre à certains de ces problèmes. Mais cela ne signifie pas qu'ils sont résolus pour autant. Cela nous apporte un panorama des points sur lesquels notre travail se devra d'être exigeant.