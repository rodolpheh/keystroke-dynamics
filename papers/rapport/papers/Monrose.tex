\section{Authentication via Keystroke Dynamics\cite{monrose1997}}

\bibentry{monrose1997}\\

%\subsection{Contexte}

%En 1997, Windows 95 est installé sur la plupart des ordinateurs personnels. Intel sort le processeur Pentium II avec une horloge allant jusqu’à 300 MHz.

%Cette même année sortent GTA et Internet Explorer 4.0. Les ordinateurs sont là depuis assez longtemps, mais leur application commence à changer grâce à l’internet

%Au moment de la publication de cet article il n’existe ni Google (1998) ni Wikipédia (2001).

\subsection{Introduction}

À l'époque de la rédaction de cette publication, Internet est encore très récent et la démocratisation de l'informatique au grand public a encore beaucoup d'avenir. L’auteur mentionne que les possibilités que donne internet semblent illimitées. Il en déduit que des nouveaux dangers sont susceptibles d'apparaître. Les mesures de sécurité courantes ne sont plus adéquates. En conséquence, de nouvelles méthodes d’authentification sont fortement demandées.\\

Selon l’auteur, la dynamique de frappe est une solution pour plusieurs raisons :\\

\begin{itemize}
\item Les résultats des observations montrent des similitudes neuro-psychologiques entre la dynamique de frappe et les signatures écrites.
\item La transparence d'utilisation pour l'utilisateur.
\item Le fait que les utilisateurs sont déjà familiers avec un clavier.
\item Le bas coût de mise en place d’une telle solution contrairement aux méthodes biométriques physiologiques.\\
\end{itemize}

L’auteur mentionne aussi que des travaux sporadiques ont déjà été menés sur le sujet.\\

\subsection{Methode experimentale}

Les données ont été collectées sur une période de sept semaines, les sujets utilisaient chacun leurs propres machines. Les sujets n’étaient pas spécialisés dans la dactylographie mais étaient tous familiers avec les ordinateurs. Les résultats étaient envoyés par courriel aux chercheurs. Environ 60\% des participants avaient conscience du but de l’expérience.

Pour la reconnaissance des profils les auteurs ont utilisé :\\

\begin{itemize}
\item La mesure euclidienne des distances entre des vecteurs n-dimensionnels ;
\item Des probabilités non pondérées ;
\item Des probabilités pondérées ;
\end{itemize}

\subsection{Résultat}

L’analyse des probabilités pondérées, comme méthode la plus efficace dans cette étude, donne une fiabilité de 90\%. Les auteurs supposent que des meilleurs résultats peuvent être atteints dans des environments plus contrôlés, comme les connexions sur les serveurs Kerberos.

\subsection{Conclusion}

Cette publication nous introduit à un petit tour historique de la dynamique de frappe, qui n'est pas une technologie nouvelle, même pour l'époque, les recherches sur le sujet ayant commencés dans les années 1980. La méthode de calcul de probabilités pondérées peut rapporter une bonne fiabilité sur des groupes de personnes limités. Néanmoins, la méthode utilisée ne prends pas en compte le rejet des profils non presents dans la base de données. C’est-à-dire, que l’algorithme peut authentifier ou rejeter uniquement les personnes ayant déjà déposé un échantillon.
