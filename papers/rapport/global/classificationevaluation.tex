\section{Classification/Évaluation}
Une fois l'acquisition des données utilisateur réalisée, il reste à produire un système de vérification capable de discriminer un imposteur par rapport à un ou plusieurs utilisateurs authentiques. Les algorithmes et méthodologies utilisées varient en fonction des objectifs recherchés.

\subsection{Méthodologies de classification}
La première étape est de construire un modèle de vérification en classant les données disponibles en plusieurs "classe" de profils. Plusieurs méthodes possibles existent pour classer les données :
\begin{itemize}
	\item
	Méthode statistique ;
	\item
	Algorithmes de \textit{Machine Learning} \bibentry{Hu2008};
	\item
	Réseaux de neurones. À proprement parler il s'agit aussi de \textit{Machine Learning} mais la littérature a tendance à les séparer du reste car ils ont des propriétés propres et requièrent des performances bien plus importantes que les algorithmes classiques.
\end{itemize}

Les méthodes utilisées varient aussi en fonction de l'objectif que l'on veut faire remplir au modèle de vérification :

\begin{itemize}
	\item une seule classe : détection des imposteurs \bibentry{killourhy2009};
	\item ne classe par utilisateur : identification de l'utilisateur parmi plusieurs \bibentry{monrose1997}.
\end{itemize}

\subsection{Métriques}
FAR / FRR

\subsection{Ordre de grandeur des systèmes développés}

Jusqu'ici les systèmes développés sont adaptés à quelques dizaines ou une centaines d'utilisateurs mais guère plus. Cependant dans la vraie vie, il n'est pas rare que les systèmes d'authentification contrôlent l'accès à des services pour des millions d'utilisateurs tous différents. Le problème est que certains algorithmes de classifications présentent une baisse très importantes de la fiabilité avec l'augmentation du nombre d'utilisateurs \cite{panasiuk2016}. Il y a donc un problème de ces méthodes de classification car elles ne permettent pas de développer des systèmes d'authentification qui sont adaptés à différents ordres de grandeur. C'est la raison pour laquelle nous avons choisi d'orienter notre travail sur l'authentification d'une personne unique sur un ordinateur personnel. En effet, l'essentiel des travaux du domaine proposent des systèmes éprouvés sur un nombre restreint d'utilisateurs, nous pourrons donc nous appuyer sur ces travaux. Par ailleurs, il est beaucoup plus difficile d'évaluer les performances d'un système développé pour des millions d'utilisateurs, notamment parce qu'il n'existe pas de bases de données permettant de l'éprouver.