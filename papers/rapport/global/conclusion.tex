\section{Conclusion et problématique}

La recherche en dynamique de frappe au clavier n'est pas récente. Cependant, après plus de quarante ans de recherche et d'optimisme, cette technique n'est toujours pas largement répandue alors que les capteurs d'empreinte digitales sont maintenant monnaie courante sur les \textit{smartphones} et les ordinateurs d'aujourd'hui. Et c'est bien malgré les attraits dont cette technique dispose \textit{a priori} : le faible coût d'acquisition et de traitement des données devrait l'avantager face à toute méthode nécessitant du matériel supplémentaire.

Outre des problèmes de propriété individuelle relevés par Peacock \textit{et al.} \cite{peacock2004}, la recherche en dynamique de frappe doit relever de nombreux défis. La qualité de la donnée et le bruit qui peut s'y trouver peuvent fortement influencer les performances du système. Il est donc difficile de développer une solution robuste quand elle dépend à ce point de la qualité de la donnée d'entrée. Et dans un contexte où les standards imposent des taux de fiabilité très importants \cite{killourhy2009}, on peut comprendre que la recherche peine à proposer des méthodologies reproductibles pour obtenir des systèmes performants en toute situation.

Il s'avère que le milieu de la recherche fait face à un problème de taille, à savoir l'accès à la donnée. Si des travaux ont tenté de normaliser cet aspect de la recherche avec des bases de données publiques de taille respectable \cite{giotGREYC,killourhy2009}, on reste encore loin de jeux de données qui permettraient de s'approcher des situations réelles de certaines applications Web gérant des millions d'utilisateurs.

Pour ces raisons, nous avons décider de nous attarder sur l'authentification d'un utilisateur sur son ordinateur personnel. Ce contexte nous permet de sélectionner les algorithmes et les méthodes les plus performants pour une quantité limitée d'utilisateurs. Des systèmes à cette échelle ont été développés de nombreuses fois, alors que les tentatives d'inclure un maximum d'utilisateurs sont rares. Choisir cette direction nous permet de nous appuyer plus solidement sur le corpus existant.

Par ailleurs, et pour correspondre à une situation réelle, nous voudrions développer une solution qui soit capable de détecter un imposteur \textbf{sans pour autant avoir été entraîné sur des données d'imposteur au préalable}. En effet, pour entraîner le modèle sur des données d'imposteur, il faudrait donner le mot de passe de l'utilisateur à plusieurs personnes afin qu'ils fassent des tentatives. Dans la vraie vie cela n'a aucun sens car il serait contre-productif de donner son mot de passe secret à quelqu'un pour renforcer son module d'authentification. Par conséquent, notre système devra proposer un moyen de détecter des imposteurs tout en ne disposant que des données de l'utilisateur authentique.

\paragraph{Problématique} : comment créer un système biométrique de dynamique de frappe au clavier en l'absence de données d'imposteurs ?