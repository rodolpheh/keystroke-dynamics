\subsection{Protocole d'acquisition}

Romain Giot \textit{et al.} \cite{giotGREYC} soulignent la difficulté de constituer une base de données conséquente pour la recherche en \textit{keystroke dynamics}. En effet, de nombreux paramètres sont susceptibles d'ajouter du bruit dans la donnée captée, voire de la rendre inutilisables. Cependant la littérature n'est pas unanime quant au fait que des erreurs de saisie viendraient renforcer la fiabilité des modèles ou la diminuer.

Romain Giot \textit{et al.} \cite{giotGREYC} ainsi que Killourhy et Maxion \cite{killourhy2009} sont ont donc tenté de répondre à cette problématique en constituant des bases de données conséquentes, basées sur un protocole d'acquisition et un formatage documentés.

Ces travaux nous permettent de nous insérer dans un ensemble de publications réutilisant ces données pour rendre les résultats comparables, ce qui n'était pas vraiment possible avant 2009.

Cependant, il est tout de même utile de souligner qu'il reste encore beaucoup à faire en la matière, notamment en comparaison des autres champs d'étude du \textit{Machine Learning}, qui disposent de bases de données de millions d'images pour la reconnaissance des visages, par exemple.

De plus, et afin de mettre en place un protocole d'acquisition de la donnée dans une application concrète, il faut avoir à l'esprit les enjeux soulignés tout au long de \bibentry{giotBenchmark}

L'état de l'art du domaine nous permet aussi d'établir qu'il existe des métriques pour évaluer le résultat de l'acquistion des données \cite{giotWeb}.