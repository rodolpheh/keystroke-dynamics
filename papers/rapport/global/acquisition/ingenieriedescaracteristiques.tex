\subsection{Ingénierie des caractéristiques}

Dans sa thèse \cite{giotThese}, Romain Giot fait l'inventaire des méthodes utilisées dans le domaine des \textit{keystroke dynamics} pour préparer la donnée avant de l'utiliser dans un modèle. Diverses méthodologies peuvent être utilisées pour formaliser les données. On a des exemples de calculs de moyennes de certaines caractéristiques de la donnée. D'autres proposent la constructions de \textit{timing vectors} \cite{killourhy2009}, soit la constitution d'un vecteur multidimensionnel dont les coordonnées correspondent à des données temporelles qui décrivent la rythmique de la frappe de l'individu.

Certains des algorithmes de \textit{Machine Learning} utilisés procèdent eux-mêmes à un traitement préalable des données pour affiner la discrimination entre les différentes classes à discriminer.