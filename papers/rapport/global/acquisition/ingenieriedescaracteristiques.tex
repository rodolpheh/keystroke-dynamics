\subsection{Ingénierie des caractéristiques}
\label{subsec:featureengineering}

Dans sa thèse \cite{giotThese}, Romain Giot fait l'inventaire des méthodes utilisées dans le domaine des \textit{keystroke dynamics} pour préparer la donnée avant de l'utiliser dans un modèle. Diverses méthodologies peuvent être utilisées pour formaliser les données. On a des exemples de calculs de moyennes de certaines caractéristiques de la donnée. D'autres proposent la constructions de \textit{timing vectors}\cite{killourhy2009}, soit la constitution d'un vecteur multidimensionnel dont les coordonnées correspondent à des données temporelles qui décrivent la rythmique de la frappe de l'individu.\\

Certains des algorithmes de \textit{machine learning} utilisés procèdent eux-mêmes à un traitement préalable des données pour affiner la discrimination entre les différentes classes à discriminer.

Mais d'une manière générale, les traitements les plus courants sur les données de frappe au clavier sont le calcul de données de frappes \textbf{calculées} depuis la donnée brute :

\begin{description}
	\item[\textit{Seek/Flight time}] : le temps entre le relâchement d'une touche et le début de la pression sur une autre touche ;
	\item[\textit{Hold/Dwell time}] : le temps entre le début de la pression d'une touche et son relâchement. \cite{kobojekRNN}
\end{description}

Souvent les données sont aussi traitées pour l'analyse des digraphes et des trigraphes (généralisable aux n-graphes \cite{Hu2008}). Il s'agit de l'étude des \textit{timings} sur des groupes de deux, trois ou n caractères. Cette technique intervient beaucoup dans la \textbf{méthodologie dynamique} \ref{subsec:ddf}. Dans ce contexte les chercheurs ont tendance à sélectionner les n-graphes au préalable en fonction de leur importance. Par exemple, le trigraphe "\textit{-ing}" sera plus intéressant à examiner pour un utilisateur anglophone que francophone du fait de la fréquence de cette séquence de lettres en Anglais.

Après avoir calculé toutes les caractéristiques jugées nécessaires au traitement, l'échantillon est mis en forme. En \textit{machine learning} cela prend la forme d'un \textbf{vecteur}, soit les coordonnées d'un point en dimension n, avec n le nombre de caractéristiques retenues, brutes ou dérivées.