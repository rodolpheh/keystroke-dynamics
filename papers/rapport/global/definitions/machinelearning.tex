\subsection{Machine Learning}

En définitive les méthodes d'identification orbitent autour de deux catégories :

* Méthodologie statistique: pour tout échantillon à évaluer, on calcule des
caractéristiques statistiques de la donnée d'entrée et on parcourt toute la base
d'échantillons enregistrés pour trouver celui qui s'en rapproche le plus ;
* Utilisation d'algorithmes de Machine Learning : pour tout échantillon à
évaluer, on le traite pour le transformer en vecteur et le modèle va le
rapprocher de la classe correspondante.

Attention : les deux méthodes ne sont pas exclusives. Après tout les algorithmes
de classification en Machine Learning sont d'abord et avant tout des méthodes
statistiques automatisées. De plus, on peut utiliser des échantillons traités au
moyen de la statistique pour "entraîner" son modèle de Machine Learning.
[Hu, Ginrich]