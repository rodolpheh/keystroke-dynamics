\subsection{Dynamique de frappe}

Les \textit{keystroke dynamics}, ou en français la \textbf{dynamique de frappe au clavier} est une méthode biométrique comportementale. Depuis le début de la recherche dans ce domaine\cite{rand}, l'intérêt de cette méthode a été de proposer une identification peu coûteuse en matériel et en traitement \cite{monrose1997}. La donnée en la matière est théoriquement abondante, et l'acquisition des données est transparente pour les utilisateurs. La littérature distingue deux types de systèmes biométriques utilisant la dynamique de frappe.\\

%\paragraph{Statique}
Identification d'un utilisateur en fonction d'un secret connu à l'avance par le modèle.
%\paragraph{Dynamique}
Identification de l'utilisateur sur la base d'un texte libre.

\begin{description}
  \item[Statique] Il s'agit de l'identification ou de l'authentification d'un utilisateur à l'aide d'un mot de passe fixe choisi par l'utilisateur (ou l'expérimentateur dans le cas des recherches). Cela permet par exemple de faire de l'authentification bi-factorielle combinant un mot de passe et la dynamique de frappe.
  \item[Dynamique] Il s'agit de l'identification ou de l'authentification d'un utilisateur sur du texte libre. Cela permet par exemple de faire du \textit{monitoring} de l'identité de l'utilisateur et de verrouiller le système lorsqu'un utilisateur non autorisé est détecté.\\
\end{description}

La dynamique de frappe au clavier va se baser sur une multitude de grandeurs mesurées pour constituer un profil qui soit unique par utilisateur\cite{giotThese,Hu2008,gunetti2005,bergadano2002}.
