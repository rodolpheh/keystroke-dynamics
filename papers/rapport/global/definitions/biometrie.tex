\subsection{Biométrie}

Wood\cite{wood1977} définie trois approches pour identifier une personne :\\

\begin{itemize}
	\item utiliser ce que l'utilisateur \textbf{sait}, c'est l'approche par mot de passe ;
	\item utiliser ce que l'utilisateur \textbf{a}, c'est l'approche retenue par les clés physiques et les systèmes double authentification par SMS ;
	\item utiliser ce que l'utilisateur \textbf{est}, c'est l'approche de la biométrie.\\
\end{itemize}

Un système de contrôle biométrique est un système automatique de mesure basé sur la reconnaissance de caractéristiques propres à l'individu. Il s'agit de l'authentification basée sur les caractéristiques physiques ou comportementales. C'est à dire ce que l'individu est ou ce qu'il sait faire.\\

Tous les systèmes biométriques génèrent, à partir de caractéristiques physiques ou comportementales, une signature qui sera ensuite comparée au modèle enregistré. Cette comparaison nous permet d'obtenir un degré de ressemblance avec le modèle enregistré. En fonction d'un seuil, défini pour que le système soit conforme à des standards de sécurité, on valide ou non l'utilisateur.

\subsubsection{L'analyse morphologique}

L'analyse des caractéristiques physiques est aussi appelée l'analyse morphologique et être appliquée sur les empreintes digitales, l'iris, la morphologie de la main ou encore les traits du visage.\\

L'analyse des empreintes digitales consiste à scanner le dessin formé par les lignes de la peau des doigts d'un individu. Les empreintes ont l'avantage d'être uniques et pratiquement immuables au cours de la vie.\\

Pour scanner l'iris, une caméra acquiert son dessin et le compare à un fichier d'identification contenant le modèle de référence stocké pour un individu.\\

La morphologie de la main est particulièrement populaire aux États-Unis. Elle consiste à mesurer des caractéristiques de la main comme la longueur et la largeur des doigts ou la forme des articulations. Le désavantage de cette méthode est la similitude des résultats pour les personnes venant de la même famille, ayant une morphologie semblable.\\

L'analyse du visage se base sur la mesure des distances entre les éléments stratégiques du visage. On analyse l'écartement des yeux,  les arêtes du nez, les commissures des lèvres, les oreilles et le menton. Les systèmes de reconnaissance du visage sont actuellement en évolution et permettent de prendre des échantillons en mouvement ou prendre en considération le vieillissement d'un individu.

\subsubsection{L'analyse comportementale}

L'analyse comportementale se concentre sur ce que l'individu sait faire de manière unique. Celle-ci peut reposer sur l'analyse de la voix, la dynamique des signatures ou la dynamique de frappes au clavier.\\

L'analyse de la voix consiste à mesurer les intonations qui sont ensuite comparées à l'empreinte pour confirmer l'identité. Ce système est utilisé pour accéder à certains service bancaires ou des lignes de service après-vente automatisées.\\

La dynamique de signature se concentre sur les gestes effectués lors de la signature. Cette analyse considère la direction et la pression d'un tracé du stylo et combine ces informations à la forme de la signature afin de déterminer l'identité.\\

La dynamique de frappe au clavier repose sur un principe similaire, mais au lieu d'observer les gestes, on mesure les temps de pression et de relâchement des différentes touches\cite{giotGREYC} ou l'ordre des n-graphes qui sont tapés\cite{bergadano2002,gunetti2005}.

\subsubsection{Analyse des performances}

Pour évaluer l'efficacité des différents systèmes biométriques nous pouvons considérer trois facteurs majeur :\\

\begin{itemize}
\item la performance - analyse des données statistiques sur le performance du système, comme EER, FTE, FTA ou temps de traitement
\item l'acceptabilité - les informations sur la perception et l'acceptation par l'utilisateur 
\item sécurité - quantifie la sécurité du système, par example combien de fraudes un imposteur peut utiliser\\
\end{itemize}

Ces trois caractéristiques doivent être prises en compte simultanément pour juger la performance d'un système biométrique. Un niveau bas de fausses identifications ne pourra pas être significatif si le taux de refus des utilisateurs légitimes est élevé. Dans se cas la solution sera condamnée à ne pas être utilisée.

\subsubsection{Conclusion}

Les méthodes d'authentification biométriques présentent des avantages de fiabilité, rapidité et facilité d'utilisation. Certains domaines peuvent encore être approfondis, notamment la dynamique de la frappe au clavier.
