\section{Lexique}

\begin{description}
  \item[DDF (Dynamique De Frappe)] Rythme de frappe au clavier
  \item[DET (Detection Error Tradeoff)] Courbe traçant le taux de faux négatifs (FRR) en fonction du taux de faux positifs (FAR).
  \item[EER (Equal Error Rate)] Mesure utilisée en biométrie lorsque le FAR et le FRR sont égaux.
  \item[Enregistrement (\textit{enrollment})] Démarche à suivre par un utilisateur pour s'enregistrer dans un système biométrique
  \item[FAR (False Acceptance Rate)] Taux de faux positifs, c'est-à-dire quand l'algorithme accepte un imposteur.
  \item[FRR (False Rejectance Rate)] Taux de faux négatifs, c'est-à-dire quand l'algorithme refuse un individu légitime .
  \item[FTA (Failure To Acquire)] Échec de la capture. Dans le cas de la dynamique de frappe, il peut s'agir d'une mauvaise saisie.
  \item[FTE (Failure To Enroll)] Échec de l'enregistrement d'un nouvel utilisateur, principalement dû à un FTA.
  \item[\textit{Keystroke Dynamics}] champ de la biométrie comportementale qui utilise les données de frappe au clavier pour reconnaître un utilisateur.
  \item[ROC (Receiver Operating Characteristic)] Courbe traçant le taux de vrais positifs en fonction du taux de faux positifs.
  \item[\textit{Timestamp}] Donnée temporelle permettant d'identifier une date avec une précision variant entre la nanoseconde et la milliseconde suivant le format.
\end{description}
